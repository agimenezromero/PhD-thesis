\thispagestyle{empty}

\begin{center}
    \textbf{\Large Summary}
\end{center}

In the face of climate change, the threat posed by vector-borne plant diseases
to global agriculture and food security has become increasingly dynamic and
unpredictable. Among these threats, Pierce's disease of grapevines, caused by
\textit{Xylella fastidiosa}, stands out due to its significant impact on
viticulture. This part of the thesis focuses on understanding and predicting
the influence of climate on the potential distribution and severity of
vector-borne plant diseases, using Pierce's disease as a case study. This has
been traditionally challenging due to the complex interactions between the
pathogen, the vector, and the host plant, as well as the influence of
environmental factors on these interactions. Species distribution models (SDMs)
have been widely used to predict the potential distribution of plant
diseases by focusing on individual components of the pathosystem, such as the
pathogen or the vector. However, these models do not yet provide the
epidemiological niche of the disease but rather the ecological niche of their
constituents. Here, we develop a mechanistic climate-driven
epidemiological model to address these limitations, integrating the effects of
climate on the vector, the pathogen, and the host plant, as well as the
interactions between these components. We validate the model using
spatio-temporal data on the distribution of Pierce's disease in the United
States, showing that it accurately captures the observed patterns. We then use
the model to predict the potential distribution of Pierce's disease under
current and future climate scenarios with available climate datasets, and
study the effect of high-resolution climate data on the model predictions. Our
results suggest that the potential distribution of Pierce's disease is
currently constrained to climatic Mediterranean regions, but  will expand
globally under future climate scenarios, with significant implications
for viticulture in Europe.

\vspace{1cm}

\begin{objectiveslist}
    \item To advance the methodologies used in modeling the potential
    distribution of vector-borne plant diseases.

    \item To predict the potential distribution of Pierce's disease under
    current and future climate scenarios with available climate datasets.

    \item To study the effect of high-resolution climate data on the
    predictions.

    \item To analyze the potential impact of Pierce's disease for viticulture
    worldwide, specially in Europe.
\end{objectiveslist}

% \vspace{1cm}

% \begin{contributionslist}
%     \item We developed a mechanistic climate-driven epidemiological model to
%     predict the potential distribution of Pierce's disease.

%     \item We validated the model using spatiotemporal data on the distribution
%     of Pierce's disease in the United States.

%     \item We predicted the potential distribution of Pierce's disease under
%     current and future climate scenarios with available climate datasets.

%     \item We studied the effect of high-resolution climate data on the model
%     predictions.

%     \item We carried out a comprehensive assessment of the potential impact of
%     Pierce's disease for global viticulture.
% \end{contributionslist}