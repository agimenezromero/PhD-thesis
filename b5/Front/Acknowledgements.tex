\begin{center}
    \textbf{\Large Agraïments}
\end{center}

{
\parindent=0pt

Aquesta tesi no hauria estat possible sense l'ajuda de moltes persones, així
que aquesta secció també serà llarga. \\

% Manuel
En primer lloc, la persona que més ha contribuit a aquest treball és el meu
director, el Dr. Manuel Matías. És gracies a ell que he tingut l'oportunitat de
realitzar aquesta tesi i començar la meva carrera investigadora. Li estic
especialment agraït per confiar en mi des del primer moment, quan el meu
historial acadèmic no era el més prometedor. També li estic agraït per la forma
en què m'ha \textbf{guiat} durant tot el procés i tot el que m'ha ensenyat,
tant en l'àmbit acadèmic com en el personal. Tinc clar que, per mi, es el
millor director que hauria pogut tenir. \\

% Colaboradors articles
Els meus colaboradors també mereixen un agraïment especial. Primer de tot vull
agrair a l'Iris Hendriks, qui va ser la primera persona amb qui vaig començar a
colaborar, ja durant el meu TFM, i Amalia Grau. Juntament amb el meu director,
conformen els coautors del meu primer article sobre l'esdeveniment de
mortalitat massiva de les nacres. Aquest treball va ser el punt de partida de
la meva tesi i va ser possible gràcies a la seva ajuda. Una menció a en
Cristóbal López i en Federico Vazquez (a.k.a Fede), amb qui vam continuar
treballant en aquest tema. Seguim amb Eduardo Moralejo, coautor de la majoria
d'articles presents en aquesta tesi, qui em va endinsar en el món de la
fitopatologia i amb qui vaig començar a treballar en el projecte de la
\textit{Xylella fastidiosa}. Aquesta col·laboració ha estat de lluny la més
fructífera i puc dir que he après moltíssim treballant amb ell. Estic molt
orgullós d'haver pogut mantenir una col·laboració realment interdisciplinària
durant aquest anys, i realment espero que segueixi en el futur. En relació amb
aquest mateix tema, també vull esmentar la Clara Lago, Arantzazu Moreno i
Alberto Fereres, amb qui he tingut l'oportunitat de col·laborar. Els estic
especialment agraïts per ``acollir-me'' durant el congrés que organitzaven a
Madrid. També agrair aquí a un científic ``de la casa'', Jose Ramasco, no només
per la part purament científica, sinó també per les converses que hem tingut.
Probablement és de les persones més ocupades del IFISC, però també de les més
maques. Per acabar amb aquest tema vull agrair a Jose Gutiérrez i Maialen
Iturbide, qui em van acollir durant la meva estada a Santander. No tinc
paraules per expressar la meva gratitud per la seva amabilitat, incloent que
m'oferissin el seu cotxe per visitar la zona! Va ser una experiència molt
enriquidora i espero que puguem seguir col·laborant en el futur. Un agraïment
també a la Susana Flecha, amb qui vam tenir una col·laboració molt fructífera.
Va ser un plaer treballar amb ella i espero que puguem seguir treballant junts.
També vull agrair a en Tomas Sintes, amb qui he tingut reunions i viatges molt
divertits. Sempre recordaré el viatje a Arabia Saudita, amb en Manuel, l'Eva i
en Miguel. Cal dir que es l'unic professor amb qui he compartit un llit! Ha
estat un plaer treballar amb ell i espero que puguem seguir col·laborant en el
futur. Finalment un agraïment especial a en Carlos Duarte, amb qui he tingut
l'oportunitat de col·laborar en un projecte molt interessant sobre coralls.
Primer de tot, agrair la seva invitació a Arabia Saudita, que com ja he dit va
ser una experiència molt enriquidora i divertida. També agrair-li per la seva
ajuda i el coneixement que m'ha transmès, encara que sigui de forma
inconscient. Espero que aquesta col·laboració segueixi en el futur. \\

% Altres col·laboradors
També vull agrair alguns col·laboradors que no han estat coautors dels articles
d'aquesta tesi, però que han contribuit de forma significativa. Primer de tot
vull agrair a en Dan Bebber, amb qui vaig tenir l'oportunitat de treballar
durant un mes en remot. També una menció especial a en Rob Salguero, amb qui
vaig tenir el plaer de treballar durant 3 mesos en una estada al departament de
biologia de la universitat d'Oxford. Va ser una experiència molt enriquidora
poder veure com es treballa en un entorn tan diferent al de l'IFISC. Aquests
tres mesos van ser també molt fructífers i vaig aprendre moltíssim. Va ser un
plaer treballar amb ell i espero que puguem seguir col·laborant en el futur.
També vull agrair a en Pere Colet i la Rosa López, amb qui he tingut
l'oportunitat de donar classes. Han confiat en mi tant per donar classes de
grau com de màster i els estic molt agraït per això. Finalment, agrair a en
Luís Gordillo, a qui vaig tenir el plaer de coneixer en persona durant la seva
visita a l'IFISC i amb qui segueixo en contacte telemàtic. \\

% IFISC
Un agraïment especial a tots els membres de l'IFISC i la UIB. En primer lloc,
a l'equip de neteja, invisibles però imprescindibles. En especial a l'Anna
Pérez, amb qui coincidia gairebé cada matí, ben d'hora. També agrair els
membres de l'administració per la seva ajuda en tot moment, sobretot a la Marta
Ozonas, l'Inma Carbonell, la Neus Lacomba, la Maria Quetglas i l'Alberto
Sánchez. També agrair a l'Adrián García tota la seva ajuda en la part de
divulgació científica. Sense ell tot el que hem fet en aquest aspecte no hauria
estat possible. Finalment, un agraiment més que especial als técnics (i
ex-técnic) de l'IFISC: l'Eduard Solivelles, l'Antònia Tugores i en Rubén
Tolosa. Literalment, sense ells aquesta tesi (i totes les de l'IFISC) no
haurien estat posibles. Els estic molt agraït per tota la seva ajuda, les seves
explicacions i la seva paciència, a més d'altres conversacions més
``informals''. \\

% Amics
També vull agrair a tots els meus amics i companys del IFISC, que han estat una
part molt important de la meva vida durant aquests anys. M'agradaría començar
amb els que fa més temps que conec. El primer lloc indiscutible l'ocupa l'Adrià
Labay (a.k.a DJ Labay), amb qui vaig començar la carrera de Física a la UAB,
amb qui casi morim al Monte Perdido (i a les festes de Martorell, les colònies
o Dresden) i sense qui, probablement, no hauria acabat la carrera. Els següents
a la llista son en David, Miguel, los canarios (Javi Galvan \& Medi) i Jogito,
qui conec des del máster. Amb ells he compartit rutillas, festes, molts
exercisis i entregues i, sobretot, moltes rialles. Una menció especial a en
David, qui també va ser un vei i un usuari del taxiUIB, i ha estat una ajuda
imprescindible en la part d'escritura d'aquesta tesi: gracies per anar dues
setmanes adelantat. També per en Miguel, amb qui he tingut converses molt
interessants i de qui admiro la seva curiositat i ganes de seguir aprenent (i
la locura de entrecot que nos hizo en su casa). Seguim amb els que vaig
coneixer entre el máster i el començament del doctorat: Pablo, Irene, Maria,
Javi Aguilar, Medea, Rodri i Mou (o Mou i Rodri). Ells em van acollir quan vaig
arribar per fer el doctorat i es van encarregar de fer la vida post-pandèmia
més fàcil i divertida. En Pablo, la Irene i la Maria mereixen una menció
especial per acompanyar-me en les meves rutes chill i no tan chill, a Sabotage
i en general aquests anys. El següent grup el formen els increibles companys i
fundadors de ZULO: Mar Ferri, Fer (desertor, pero tt gym), Mar Cuevas, Gorka,
Guillem, Pau i Manuel Miranda. Aquests 10 metres quadrats sense llum natural ni
ventilació haurien estat insuportables sense vosaltres. Aqui també agrair les
noves incorporacions, com en Jaume, la Sara o en Coque. Finalment tenim alguns
dels integrants de la S07: Jesus, Bea, Lisa, Pepe, Dimitris i Daniele. Una
menció especial a Jesus i Bea, per haver compartit mes temps amb mi i, en
conseqüència, haver aguantat més chapes i gaudit de més barbacoas amb tremendo
allioli. També voldria agrair als postdocs amb qui hem comparitit dinars i
altres moments, sobretor l'Eva. Finalment, vull tornar a agrair en Manuel
Matías, a qui també considero un amic. Espero que, a part de seguir
col·laborant, puguem seguir compartint sopars i copes de vi. \\

% Barri, Carmela, família
Per acabar, vull agrair als meus amics més antics. Als companys i amics de la
carrera: Bernat, Roger, Uri, Linde, Jota i Dani. Tot i que no ens haguem vist
gaire, ni jo parli massa pel grup, sempre heu estat aquí. En especial agrair a
en Dani, qui sempre ha estat disponible per fer una birra i uns bons nachos al
Livingstone. També agrair als amics del barri, especialment a la Lidia, el
Víctor, l'Alex Jimenez i el Guti. Fora de l'ambit acadèmic, son els amics que
més han escoltat tant els meus exits i avanços com les meves queixes,
frustracions i fracassos. També agrair a la Carmela, que ha estat una part molt
important de la meva vida durant aquests anys. Ella es l'única persona que ha
estat al meu costat les 24 hores del dia, 7 dies a la setmana, 365 dies a
l'any. Sense ella segurament hauria acabat aquesta tesi, pero probablement
també hauria acabat amb la meva salut mental. Tinc molta sort de tenir-la al
meu costat, des de Barcelona, a Mallorca, a Oxford i properament a Blanes; a
tot arreu on de moment m'ha seguit. Li estic especialment agraït. També he de
dir que si no fos per ella moltes figures d'aquesta tesi serien apreciablement
més lletges... Finalment vull agrair a la meva família per haver-me donat
l'oportunitat de formar-me i per haver-me donat suport en tot moment.
}

\vfill