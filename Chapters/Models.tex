\section{\label{sec:Introduction} Introduction}

- Modeling human dynamics in opinion formation processes, or technology adoption process has been object of study for many years. 

- The contagion of ideas is a process that has been studied for many years, and it is a process that is present in many social systems.

\dots

\section{\label{sec:Simple and Complex Contagion} Simple and Complex Contagion}

- The contagion of ideas is a process that has been studied for many years, and it is a process that is present in many social systems. 

- Simple contagion is used to describe the process of contagion of ideas that is driven by the number of contacts that an individual has with other individuals.

- Complex contagion is used to describe the process of contagion of ideas that is driven by the number of contacts that an individual has with other individuals, and by the number of contacts that the individuals that the individual has contact with have with other individuals.

\section{\label{sec:Threshold model} Threshold model}

- The threshold model is used to describe processes of complex contagion based on the idea that individuals have a threshold that needs to be reached in order to adopt an idea.

- Watts (2002) proposed a simple model to study the dynamics of complex contagion based on the idea that individuals have a threshold that needs to be reached in order to adopt an idea. 

- Should I describe what is a phase transition? Maybe with a foot note.

- This model exhibits a phase transition from a regime where the adoption of an idea is rare to a regime where the adoption of an idea is common. This phase transition is discontinuous.

- This model has been studied for many social networks and many variants of the model have been proposed\dots

\section{\label{The Sakoda-Schelling model} The Sakoda-Schelling model}

- Thomas C. Schelling (1975) proposed a model to study the dynamics of segregation based on the idea that individuals have a threshold that needs to be reached in order to move to a different location. 

- This model become very popular because it was able to reproduce the segregation patterns that are observed in many cities, just as an emergent phenomena from individual decisions.

- On the other hand, this model was already studied by Sakoda in 1971, in a paper that was published in Japanese, and that was not known by the scientific community. In this work, the author proposed a model to study the dynamics of segregation based on the idea that individuals have a threshold that needs to be reached in order to move to a different location.

- Schelling model is a particular case of the Sakoda model.

- This model exhibits a phase transition from a regime where the segregation is low to a regime where the segregation is high. This phase transition is discontinuous. 

- Despite the many variants of the model that have been proposed, the phase transition emergent in this model is very robust.

\section{\label{sec: Bursty Human Dynamics} Bursty Human Dynamics}

- All models previously described are based in an important assumption regarding to its dynamics: the interactions between individuals occur at a constant rate.

- This constant rate assumption is assuming that the stochastic interactions follow a Poisson process. 

- However, from the analysis of many datasets of human interactions, it has been observed that the interactions between individuals are not constant, but bursty.

- The bursty behaviour is characterized by power law interevent time distributions, and it is present in many human activities, such as e-mail communication, face-to-face interactions, and phone calls.

- The modeling of bursty human dynamics is important because it allows us to understand how the bursty behaviour of human interactions affects the dynamics of social systems.

- Previous literature has several approaches to model bursty human dynamics: temporal networks, activity-driven models, aging \dots

\section{\label{sec:Aging mechanism} Aging mechanism}

- The aging mechanism is a mechanism that has been proposed to model the bursty behaviour of human interactions.

- Aging mechanism is based on the idea that the probability of an individual to interact with another individual decreases with the time since the last interaction.

- Attachment to previous beliefs or habits is a common feature in human behavior. Granovetter (1973) discussed 

- Instead of the constant rate assumption, the aging mechanism assumes that the interactions between individuals occur at a rate that decreases with the time since the last interaction.

- The focus on this approach is to include the bursty dynamics in the individuals attempts to interact with others, rather than in the interactions themselves (as in the activity-driven models or temporal networks).

\section{\label{sec:Aging in Simple Contagion Models} Aging in Simple Contagion Models}

- Aging in the Voter model has been studied by many authors, and it has been shown that the aging mechanism affects the dynamics of the model.

- The aging mechanism has been shown to affect the dynamics of the Voter model, and to change the phase transition of the model.

- Aging in the noisy voter model is able to change the phase transition of the model, and to make the phase transition continuous.

- Aging in the SI model is able to change the cascade size distribution of the model, and to make the cascade size distribution follow a power law (see the work of Karsai et al. 2011).