
\section{\label{sec:Infectious disease modelling} Infectious
  disease modelling}

Infectious diseases are caused by pathogenic microorganisms, such as
bacteria, viruses, parasites, or fungi, when they invade other organisms, which
are called \textit{hosts} in this context. These pathogens proliferate inside
the host and can disrupt the normal functioning of the host organism by
damaging tissues, altering physiological processes, or triggering immune
responses that contribute to illness. Infectious diseases can affect a wide
variety of organisms, from plants and animals to humans, and they can spread
through various means, including direct interaction from infected individuals,
both by close contact or at a distance, or through vectors. Vectors are
organisms that transmit pathogens from one host to another, such as mosquitoes,
that usually are not affected by the disease themselves.

Modelling the spread of infectious diseases has a long history, with the first
mathematical model coming from the hand of Bernoulli in 1760
\cite{Bernoulli1760}, who developed a model to understand the spread of
smallpox. However, it was not until the early 20th century that we find the
foundation of the modern theory of infectious disease dynamics, with the
seminal work of Ronald Ross and Hilda P. Hudson
\cite{Ross1916,Ross1917,Ross1917_2} and Kermack and McKendrick
\cite{McKendrick}. Since then, a wide range of models have been
developed to understand the dynamics of infectious diseases and to predict
their spread. In general, these models can be classified into two main
categories: compartmental models and individual-based models.

\subsection{\label{sec:Compartmental models} Compartmental models}

Compartmental models are based on the assumption that the population can be
divided into different compartments or categories, each representing a
different state of the disease (e.g. susceptible to infection and infected).
Individuals move from one compartment to another following some dynamical
rules. Under the assumption of a well-mixed and sufficiently large population,
one can consider that  every pair of individuals has equal probability of
coming into contact with one another and that fluctuations in the number of
individuals in each compartment can be neglected. This is known as the mean
field approximation. Under these assumptions the dynamics of the disease can be
described by a set of differential equations that govern the transitions
between compartments.

\subsubsection*{\label{sec:The SIR model} The SIR model}

The most famous compartmental model is the SIR model, which is a simple
particular case of the general mathematical framework formulated by Kermack and
McKendrick \cite{McKendrick}. Because the original Kermack-McKendrick model
is quite general and complex, I will explain the SIR model following a more
modern and intuitive approach. The SIR model divides the population into three
compartments: susceptible individuals (S), infected individuals (I), and
recovered individuals (R). Individuals come into contact with each other at a
given rate $a$ and, in the case of a $S-I$ contact, the susceptible individual
becomes infected with a probability $b$. We assume that the incubation period
is short enough to be negligible; that is, a susceptible who contracts the
disease is infective right away. Infected individuals recover at a constant
rate $\gamma$, and once recovered, are assumed to be immune to the disease and
cannot be reinfected. We finally assume that there is no entry into or
departure from the population, no birth or natural deaths, so that
the population remains constant.

According to the mean field approximation, the probability that an infected
individual contacts a susceptible one is given by $a\cdot S/N$. Thus, the
average number of contacts between infected and susceptible individuals is
given by $a\cdot S/N\cdot I=a\cdot SI/N$. Finally, as the probability that a
susceptible individual becomes infected after a contact is $b$, the average
number of new infections will be given by the product of this probability and
average number of contacts between infected and susceptible individuals,
$b\cdot a\cdot SI/N=\beta SI/N$. These considerations lead to the following
description of the model given by a system of differential equations,
\begin{equation}\label{eq: normal_SIR_theo}
  \begin{array}{l}
    \dot{S}=-\beta SI/N          \\
    \dot{I}=\beta SI/N -\gamma I \\
    \dot{R}=\gamma I \ ,
  \end{array}
\end{equation}

Despite the aparent simplicity of the model, the non-linear nature of the
differential equations makes it difficult to find analytical solutions.
However, important quantitative insights can still be obtained by performing an
analytical study of the model. We will now derive some important results from
the SIR model, most of which cane be already found in reference
\cite{Murray_book}.

First let's start by the simplest insight: the assumption of population
conservation is inherently included in the model. We can prove this statement
by just adding the three differential equations in \cref{eq: normal_SIR_theo},
\begin{equation}
  \dot{S}+\dot{I}+\dot{R}=0\Longrightarrow S+I+R=\mathrm{const.}=N \ .
\end{equation}

Physicists usually call this kind of quantity a \textbf{conserved quantity}
or a \textbf{conservation law}. In this case, the conservation law has a
particular meaning: the total number of individuals in the population remains
constant, which was already assumed in the model. In general, conservation laws
can arise from other symmetries of the system and are not always so obvious. In
any case, these conserved quantities allow us to reduce the number of
independent variables in the system, which can be very useful to simplify the
analysis of the model. In this case, the conservation law allows us to reduce
the number of independent variables from three to two, so that we can study the
dynamics of the system in a two-dimensional phase space given by the variables
$S$ and $I$, as $R$ can be obtained from the relation $R=N-S-I$.

Now consider the starting point of an epidemic, such our old friend the
COVID-19 pandemic. For sure at the beginning of the epidemic the number of
infectedindividuals is not zero, $I(0)>0$, while the number of recovered
individuals is indeed zero, $R(0)=0$. Thus, the number of susceptible
individuals at the beginning of the epidemic is $S(0)=N-I(0)$. If an epidemic
is to develop, the number of infected individuals must increase at the
beginning of the epidemic, but which are the conditions for this to happen?

Just by considering the differential equation for $I$ in \cref{eq:
  normal_SIR_theo} with our initial conditions we find the following
expression,
\begin{equation}\label{eq: threshold}
  \der{I}{t}\Big|\limitss{t=0}{}=I(0)\parentesi{\beta
    S(0)/N-\gamma}\Longrightarrow\left\{\begin{array}{l}
    \displaystyle\der{I}{t}\Big|\limitss{t=0}{}<0 \Longleftrightarrow
    S(0)<\frac{\gamma N}{\beta} \\\\
    \displaystyle\der{I}{t}\Big|\limitss{t=0}{}>0 \Longleftrightarrow
    S(0)>\frac{\gamma N}{\beta}
  \end{array}\right. \ \ .
\end{equation}

The condition \cref{eq: threshold} defines a threshold for the developing of an
epidemic in the $SIR$ model! This is, there is a critical number of susceptible
individuals below which the epidemic will not develop,
\begin{equation}
  S_c=\frac{\gamma N}{\beta}\equiv \rho \ .
\end{equation}

The critical parameter $\rho$ is sometimes called the \textit{relative removal
  rate} and its reciprocal, $\sigma=\beta/(\gamma N)$, the \textit{infection’s
  contact rate}. This threshold behaviour can be further simplified by
considering the so-called \textit{basic reproduction number}, $R_0$, defined
as,
\begin{equation}
  R_0=\frac{S(0)}{\rho}=S(0)\sigma=\frac{\beta S(0)}{\gamma N} \ ,
\end{equation}
which measures the number of secondary infections given by a primary
infection in a whole susceptible population. I am sure you have already heard
about the basic reproduction number in the context of the COVID-19 pandemic.

To see that this quantity measures the number of secondary infections produced
by a primary infection we just need to recall the definition of the rates and
``read'' the expression. $\beta S(0)/N$ is the rate of new infections produced
by a primary infection (as we are at time $t=0$). In other words, $\beta
  S(0)/N$ is the number of secondary infections produced by a primary infection
\textit{per unit time}. Finally, we observe that, as $\gamma$ is the rate of
recovery of infected individuals, $1/\gamma$ is the average time an
individual remains infected. The product of the number of secondary infections
produced by a primary infection per unit time and the average time an infected
individual remains infected gives the total number of secondary infections
produced by a primary infection in the whole susceptible population. Voilà!
This is nothing but the basic reproduction number, $R_0$.

It can be formally proved that this quantity defines a threshold for the
development of an epidemic from the fact that $S$ is a monotonically decreasing
function, which implies $S(t)<S(0) \ \forall t>0$.
\begin{equation}\label{eq: threshold_t}
  \der{I}{t}=I\parentesi{\beta S-\gamma}\leq I\parentesi{\beta
    S(0)-\gamma}=\gamma I\parentesi{\frac{S(0)}{\rho}-1}=\gamma
  I\parentesi{R_0-1} \
  ,
\end{equation}
so that for $R_0<1$, $\dot{I}<0 \ \forall t>0$ and thus $I(0)>I(t)$ as
$t\to\infty$, which basically means that the epidemic dies out, while for
$R_0>1$ the epidemic grows.

This threshold behaviour agrees with our intuition given the definition of
the basic reproduction number: if a primary infection produces more than 1
secondary infection an epidemic will develop, while if it does not reach to
infect at least one individual it will die out. Furthermore, note that
because of our mean field approach, the number of secondary infections produced
by a primary one refers to the \textit{average} number of secondary
infections.

Another important analytical result that can be obtained from this model is
the maximum of infected individuals, that gives an idea of how severe the
epidemic will be. To find this maximum we just need to find the maximum of
$I(t)$, which is given by the condition $\dot{I}=0$. From the differential
equation for $I$ in \cref{eq: normal_SIR_theo} we find the following relation,
\begin{equation}
  \der{I}{t}=0\Longrightarrow\beta S/N-\gamma=0\Longrightarrow
  S=\frac{\gamma N}{\beta}=\rho \ .
\end{equation}

So the maximum of infected individuals, given the development of a proper
epidemic, will take place when $S(t)=\rho$. But we still don't know the
maximum number of infected individuals. To do so, we first need to go through
a smart mathematical trick. Dividing the differential equations for $S$ and $I$
(considering $I\neq0)$) we obtain,
\begin{equation*}
  \frac{\dif I}{\dif S}=-1+\frac{\gamma}{\beta S}=-1+\rho/S \quad (I\neq0) \
  .
\end{equation*}

Integrating the relation,
\begin{equation*}
  \int_{I(0)}^{I}\dif I=\int_{S(0)}^S\dif S \keys{-1 + \rho/S} \Longrightarrow
  I-I(0)=S(0)-S+\rho\ln(\frac{S}{S(0)}) \ ,
\end{equation*}
we obtain the phase plane trajectories for $S$ and $I$ given by,
\begin{equation}\label{eq: max_I_eq}
  I+S=I_0+S_0+\rho\ln(\frac{S}{S(0)})=N+\rho\ln(\frac{S}{S(0)}) \ ,
\end{equation}
where we have considered $I(0)+S(0)=N$ given that $R(0)=0$.

Note that \cref{eq: max_I_eq} can be rewritten as
$I+S-\rho\ln(S)=N-\rho\ln(S_0)=\mathcal{C}$, so that the right-hand side of the
equation is a constant. This means that we have just found another conservation
law for the system! This one is not so obvious as the previous one, but it is
still a very useful result. In essence, we now can describe the dynamics of the
system with only one independent variable, $S$, as $I$ can be obtained from the
relation in \cref{eq: max_I_eq}.

Finally we find an expression for the maximum of infected individuals by
substituting the condition previously found (the maximum occurs when
$S(t)=\rho$)
in \cref{eq:  max_I_eq},
\begin{equation}
  I(t)=N+\rho\ln{\frac{S(t)}{S(0)}}-S(t) \Longrightarrow
  I_{max}=N+\rho\claudator{\ln(\frac{\rho}{S_0}) - 1} \ .
\end{equation}

Similarly we can find the final number of susceptible individuals at the end
of
the epidemic, for sure a very important quantity. By dividing the
differential
equations for $S$ and $R$, we obtain,
\begin{equation*}
  \frac{\dif S}{\dif R} = -\frac{\beta}{\gamma}S=-\frac{S}{\rho}
  \Longrightarrow\int_{S(0)}^S\frac{\dif
    S}{S}=-\frac{1}{\rho}\int_{R(0)=0}^R\dif
  R\Longrightarrow\ln(\frac{S}{S(0)})=-\frac{R}{\rho} \ .
\end{equation*}

As $I(\infty)=0$, necessarily $R(\infty)=N-S(\infty)$ so that we get the
following expression for the final number of susceptible individuals,
\begin{equation}\label{eq: trascendental_S_inf}
  S(\infty) - \rho\ln(S(\infty))=N-\rho\ln(S(0)) \ .
\end{equation}

$S(\infty)$ is nothing but the positive root of the transcendental equation
\cref{eq: trascendental_S_inf}. In fact, this transcendental equation can be
solved by means of the Lambert's $W$ function \cite{Lethonen2016},
\begin{equation}
  S(\infty)=-\rho\cdot W_0\claudator{-\frac{S(0)}{\rho} \,
    e^{-N/\rho}}=-\rho\cdot W_0\claudator{-R_0 \, e^{-N/\rho}} \ .
\end{equation}

% The Lambert's $W_l(x)$ function is bivalued for $x\in(-1/e,0)$, so that it is
% necessary to choose a branch for such values of $x$. In this epidemiological
% context it has been already shown that the branch $l=0$ has to be chosen when
% $S(t)<S_c$ \cite{Rodri_thesis}, and clearly $S_\infty<S_c$. With this analysis
% we ensure that the final number of susceptible individuals is a strictly
% positive number, $S_\infty>0$, given the properties of the Lambert's
% function.

So far we have shown how, under minimal assumptions, one can build an epidemic
model such as the SIR model. In addition, we have shown how to obtain some
important analytical results from the model, such as the threshold for the
development of an epidemic, i.e. the \textit{basic reproduction number}; the
maximum number of infected individuals, and the final number of susceptible
individuals.

However, the SIR model is one of the simplest compartmental models, and more
complex models have been developed to account for more realistic scenarios. For
such models, analytical results are usually not available, and numerical
simulations are required to understand the dynamics of the diseases they
describe. Fortunately, there are a couple of formal approaches that can be used
to derive the basic reproduction number of basically any compartmental model:
linear stability analysis and the next generation matrix approach.

\subsubsection*{Linear stability analysis}

Linear stability analysis is a powerful tool that can be used to study the
dynamics of a \textbf{dynamical system} around its \textbf{fixed points}. The
basic idea is to linearize the system of differential equations around these
equilibrium points and study the stability of the system by analysing the
\textbf{eigenvalues} of the \textbf{Jacobian} of the system. A
\textit{dynamical system} is a system that evolves over time,
such as the SIR model. The \textit{fixed points} of the system are the points
where it does not change over time, i.e. the points where the derivatives of
the variables of the model are zero. The \textit{Jacobian} of the system is a
matrix that contains the first-order partial derivatives of the variables of
the system. The \textit{eigenvalues} of the Jacobian matrix are the roots of
the characteristic polynomial of the matrix, obtained through the equation
$\det(J-\lambda I)=0 $, and they determine the stability of the system. If the
real part of all the eigenvalues is negative, the system is stable, while if
the real part of at least one eigenvalue is positive, the system is unstable.
Let's see how this works in practice.

For the SIR model, the fixed points are given by the condition
$\dot{S}=\dot{I}=\dot{R}=0$, which implies $I=0$ with any value of $S$ and $R$.
Thus, the fixed points of the system are given by $(S^*,I^*,R^*)=(S,0,N-S)$.
These fixed points are also called \textit{disease-free} state of the system,
as the number of infected individuals is zero. The Jacobian matrix of the
system is given by,
\begin{equation}
  J=\begin{pmatrix}
    \partial\dot{S}/\partial S & \partial\dot{S}/\partial I &
    \partial\dot{S}/\partial R                                \\
    \partial\dot{I}/\partial S & \partial\dot{I}/\partial I &
    \partial\dot{I}/\partial R                                \\
    \partial\dot{R}/\partial S & \partial\dot{R}/\partial I &
    \partial\dot{R}/\partial R
  \end{pmatrix}=\begin{pmatrix}
    -\beta I/N & -\beta S/N       & 0 \\
    \beta I/N  & \beta S/N-\gamma & 0 \\
    0          & \gamma           & 0
  \end{pmatrix} \ ,
\end{equation}

Now we substitute the fixed point in the Jacobian matrix and we obtain,
\begin{equation}
  J=\begin{pmatrix}
    0 & -\beta S^*/N       & 0 \\
    0 & \beta S^*/N-\gamma & 0 \\
    0 & \gamma             & 0
  \end{pmatrix} \ .
\end{equation}

The eigenvalues of the Jacobian matrix are the roots of the characteristic
polynomial of the matrix, which is given by $\det(J-\lambda I)=0$. In this case
the characteristic polynomial is given by,
\begin{equation}
  \det(J-\lambda I)=\begin{vmatrix}
    -\lambda & -\beta S^*/N                & 0        \\
    0        & \beta S^*/N-\gamma -\lambda & 0        \\
    0        & \gamma                      & -\lambda
  \end{vmatrix}=\lambda^2(\beta S^*/N-\gamma-\lambda) \ .
\end{equation}

The roots of the characteristic polynomial are given by $\lambda_1=\lambda_2=0$
and $\lambda_3=\beta S^*/N-\gamma$. As we have previously stated, the stability
of the system is determined by the real part of these eigenvalues. Of course,
the eigenvalues $\lambda_1$ and $\lambda_2$ are zero, so they do not provide
any information about the stability of the system.

The eigenvalue $\lambda_3$ is negative if $\beta S^*/N<\gamma$, thus meaning
that the fixed point $(S^*,I^*,R^*)$ is stable if this condition is satisfied.
On the other hand, the fixed point is unstable if $\beta S^*/N>\gamma$. Indeed,
this conditon defines the basic reproduction number, $R_0$, as we can rewrite
the condition for the stability of the fixed point as $R_0=\beta S^*/\gamma N$
greater or smaller than 1.

% Comment on the conserved quantities that can be already observed from the Jacobian matrix.
Finally note that the Jacobian matrix also provides us with another important
insight about the system: the presence of conserved quantities. In this case,
the sum of the first and third rows of the Jacobian matrix is zero, which
indicates that the system has two conserved quantities, which we have already
found: the total number of individuals in the population and \cref{eq:
  max_I_eq}.

\begin{remark}
  The stability of the fixed points of a dynamical system can be determined by
  analysing the eigenvalues of the Jacobian matrix of the system. If the real
  part of all the eigenvalues is negative, the fixed point is stable, while if
  the real part of at least one eigenvalue is positive, the fixed point is
  unstable. In epidemic models the basic reproduction number, $R_0$, determines
  the	stability of the disease-free states of the system, which are the fixed
  points where the number of infected individuals is zero. If $R_0<1$, the
  fixed points are stable, while if $R_0>1$, the fixed points are unstable.
  The basic reproduction number is a threshold for the development of an
  epidemic: if $R_0<1$, the epidemic will die out, while if $R_0>1$, the
  epidemic will propagate. In addition, the Jacobian matrix of the system
  provides us with important insights about the system, such as the presence of
  conserved quantities.
\end{remark}

\subsubsection*{The next generation matrix method}
We have previously shown that the basic reproduction number, $R_0$ of a
compartmental model can be obtained by analysing the stability of the fixed
points corresponding to the disease-free state of the model. Specifically, the
basic reproduction number, $R_0$, is related to the largest non-zero eigenvalue
of a fixed point, $\Lambda$, such that $R_0>1$ if $\Lambda>0$. However, the
Jacobian matrix of a compartmental model can be quite large and complicated, so
the derivation of the basic reproduction number following this method can be
cumbersome. The next generation matrix method is an ingenious method that can
be used to derive the basic reproduction number of any compartmental model
directly, without the need to analyse the stability of the fixed points.

In the NGM method, $R_0$ is identified as the dominant eigenvalue of a suitably
defined linear operator (a linear matrix in a suitable basis). This operator is
obtained by decomposing the Jacobian of the infected/infecting compartments
evaluated at the disease-free state,  $J^*$, in the form $J=T+\Sigma$, where
$T$ is the \textit{transmission part}, that describes the production of new
infections, and $\Sigma$  the \textit{transition part}, that describes changes
of state. Then, it can be proved \cite{Diekmann2010} that the \textit{basic
  reproduction number} $R_0$ is given by the spectral radius (i.e. the largest
eigenvalue) of the next generation matrix, $K$, defined as,
\begin{equation}
  R_0=\rho(K) \quad \textrm{with} \quad K=-T\Sigma^{-1} \
\end{equation}

To appreciate the power of the NGM method, let's see how it can be applied to a
slightly more complex model than the SIR model. We will now consider a
compartmental model for vector-borne plant diseases.

\subsubsection*{A compartmental model for vector-borne plant diseases}

The simplest way to do this is to consider a compartmental model with five
compartments: susceptible individuals ($S$), infected individuals ($I$),
recovered individuals ($R$), susceptible vectors ($S_v$), and infected vectors
($I_v$). For most vector-borne plant diseases, we can assume that the only
mechanism of disease spread is the direct transmission from infected vectors to
susceptible plants at a given rate $\alpha$. Infected individuals recover at
a rate $\gamma$. The total number of plants is given by $N$, and the total
number of vectors is given by $N_v$. Vectors are assumed to be born at a
constant rate $\delta$ proportional to the total number of vectors and die at a
constant rate $\mu$. With these assumptions, the dynamics of the disease can be
described by the following system of differential equations,
\begin{equation}\label{eq: SIRV_model}
  \begin{array}{l}
    \dot{S}=-\beta SI_v/N_v                       \\
    \dot{I}=\beta SI_v/N_v -\gamma I              \\
    \dot{R}=\gamma I                              \\
    \dot{S_v}=\delta N_v -\alpha S_vI/N - \mu S_v \\
    \dot{I_v}=\alpha S_vI/N-\mu I_v \ ,
  \end{array}
\end{equation}

We can easily check that the plant population is conserved by adding the
differential equations for $S$, $I$, and $R$,
\begin{equation}
  \dot{S}+\dot{I}+\dot{R}=0\Longrightarrow S+I+R=N \ .
\end{equation}

Similarly, to check if the vector population is conserved we add the
differential equations for $S_v$ and $I_v$,
\begin{equation}
  \dot{S_v}+\dot{I_v}=\delta N_v -\mu S_v -\mu I_v=0\Longrightarrow
  N_v(\delta-\mu)=0 \ .
\end{equation}

This equation implies that the vector population is conserved only if
$\delta=\mu$, which is a reasonable assumption given that the birth and death
rates of vectors are usually similar. By now, this is the case we will
consider in the following analysis, but we will see in \cref{part:VBD}
that thinks can get really complicated if we relax this assumption.

After this initial check, we can proceed to obtain an expression for the basic
reproduction number, $R_0$. To compute it we can follow the next generation
matrix approach. We first need to identify the disease-free state of the
system, which is given by $(S^*,I^*,R^*,S_v^*,I_v^*)=(S(0),0,N-S(0),N_v,0)$.
Then we need to write down the Jacobian matrix of the subsystem corresponding
to the infected/infecting compartments, $I$ and $I_v$ in this case and evaluate
it at the disease-free state. This matrix is given by,
\begin{equation}
  J=\begin{pmatrix}
    \partial \dot{I}/\partial I   & \partial \dot{I}/\partial I_v   \\
    \partial \dot{I_v}/\partial I & \partial \dot{I_v}/\partial I_v
  \end{pmatrix}=
  \begin{pmatrix}
    -\gamma      & \beta S/N_v \\
    \alpha S_v/N & -\mu
  \end{pmatrix}=
  \begin{pmatrix}
    -\gamma      & \beta S(0)/N_v \\
    \alpha N_v/N & 0
  \end{pmatrix} \ ,
\end{equation}
where we have substituted the fixed point
$(S^*,I^*,R^*,S_v^*,I_v^*)=(S(0),0,N-S(0),N_v,0)$ in the last step.

Next we decompose the matrix in the transmission and transition parts,
$J=T+\Sigma$, and compute the inverse of the transition part, $\Sigma^{-1}$,
\begin{equation}
  T=\begin{pmatrix}
    0 & \beta S(0)/N_v \\
    0 & 0
  \end{pmatrix} \quad \text{and} \quad
  \Sigma=\begin{pmatrix}
    -\gamma      & 0    \\
    \alpha N_v/N & -\mu
  \end{pmatrix} \quad \Longrightarrow \quad
  \Sigma^{-1}=\frac{-1}{\gamma\mu}\begin{pmatrix}
    \mu          & 0      \\
    \alpha N_v/N & \gamma
  \end{pmatrix} \ .
\end{equation}

We finally obtain the next generation matrix, $K$, as the product of the
transmission and transition parts,
\begin{equation}
  K=-T\Sigma^{-1}=\frac{1}{\gamma}\begin{pmatrix}
    \frac{\beta\alpha}{\gamma\mu}\frac{S(0)}{N} &
    \frac{\beta}{\mu}\frac{S(0)}{N}                 \\
    0                                           & 0
  \end{pmatrix} \ .
\end{equation}

The basic reproduction number, $R_0$, is given by the spectral radius of the
next generation matrix, $K$, which is the largest eigenvalue of the matrix. In
this case, the matrix is a $2\times2$ matrix, so the eigenvalues can be easily
obtained. The characteristic polynomial of the matrix is given by,
\begin{equation}
  \det(K-\lambda I)=\begin{vmatrix}
    \frac{\beta\alpha}{\gamma\mu}\frac{S(0)}{N}-\lambda &
    \frac{\beta}{\mu}\frac{S(0)}{N}                                \\
    0                                                   & -\lambda

  \end{vmatrix}=\lambda\parentesi{\lambda-\frac{\beta\alpha}{\gamma\mu}
    \frac{S(0)}{N}}
  \ ,
\end{equation}
so the eigenvalues are $\lambda_1=0$ and $\lambda_2=\beta\alpha S(0)/(\gamma\mu
  N)$. The basic reproduction number, $R_0$, is given by the largest eigenvalue
of the matrix, so
\begin{equation}
  R_0=\frac{\beta\alpha S(0)}{\gamma\mu N} \ .
\end{equation}

\subsection{\label{sec:Individual-based models} Individual-based models}

Individual-based models describe the dynamics of infectious diseases by
considering the individuals of the population as discrete entities that can
interact with each other. These models are particularly useful when the
population is not well-mixed, the interactions between individuals are
heterogeneous, the individuals can not be considered as identical or when an
spatial setting is considered. In individual-based models, the dynamics of the
disease are described by stochastic processes that govern the interactions
between individuals. These models are usually more complex than compartmental
models, so that analytical results are not usually available, but they can
provide more realistic insights into the dynamics of infectious diseases. Thus,
to study the dynamics of individual-based models, we must rely on numerical
simulations.

There are several ways to simulate the dynamics of individual-based models,
but one of the most common methods is the Gillespie algorithm, which is indeed
an exact and unbiased algorithm to simulate stochastic processes. The Gillespie
algorithm is based on the idea of simulating the dynamics of the system by
considering the possible events that can occur and the rates at which they
occur. The algorithm proceeds by selecting one of the possible events at random
and updating the state of the system accordingly. This process is repeated
until the desired time is reached or the system reaches a steady state.

\section{\label{sec:Disease biogeography} Disease biogeography}

% Ecological niche concept
\subsection{\label{sec:Ecological niches} Ecological niches}

% Explaining the concept of ecological niches

% Species distribution models
\subsection{\label{sec:Species distribution models} Species distribution
  models}

\subsection{\label{sec:The role of climate change} The role of climate change}

\section{\label{sec:Data-driven modelling} Data-driven modelling}

\subsection{\label{sec:Climate data} Climate data}

\subsection{\label{sec:Remote sensing} Remote sensing}

\subsection{\label{sec:Machine learning} Machine learning}

\section{\label{sec:Case studies} Case studies}

\subsection{\label{sec:The Mass Mortality Event of Pinna nobilis} The Mass
  Mortality Event of \textit{Pinna nobilis}}

\subsection{\label{sec:Xylella fastidiosa: an emerging global
    threat}\textit{Xylella
    fastidiosa}: an emerging global threat}

\subsection{\label{sec:Coral reefs under threat} Coral reefs under threat}

\subsection{\label{sec:Ocean acidification} Ocean acidification}

\subsection{\label{sec:The decline of seagrass meadows} The decline of seagrass
  meadows}