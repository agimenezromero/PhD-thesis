\thispagestyle{empty}

\begin{center}
    \textbf{\Large Summary}
\end{center}

Vector-borne plant diseases are a significant threat to agriculture, with the
potential to cause widespread epidemics, food shortages and economic losses.
In particular, the bacterium \textit{Xylella fastidiosa} has emerged as a major
concern for the agricultural sector, affecting a wide range of crops worldwide.
Despite extensive research, many aspects of the dynamics of vector-borne plant
diseases remain poorly understood. For instance, the dynamics of the vector
population, which play a crucial role in disease transmission, are often
neglected in existing models. This is particularly important for
\textit{Xylella fastidiosa} diseases, where the vector population exhibits
complex non-stationary dynamics that are not captured by traditional
models. In this part we focus on the modelling of vector-borne plant disease in
which the vector population follows complex non-stationary dynamics. We develop
a theoretical framework for modelling vector-borne diseases with non-stationary
and non-periodic vector populations. We show that traditional methods to
predict the onset of an epidemic do not apply in this context, propose new
approaches and demonstrate that these dynamics can have a significant
effect on the temporal patterns of disease spread, leading to unexpected
outcomes that are not captured by traditional models. This foundational work
enabled the construction of a model for \textit{Xylella fastidiosa} diseases
that explicitly incorporates the complex dynamics of the vector population. We
validate the model using empirical data, demonstrating its predictive power and
practical utility. Finally, we provide insights into the design of effective
control strategies that take into account the dynamics of the vector
population. We address current gaps in our understanding of how non-stationary
vector dynamics influence disease spread and severity, offering new insights
into the management and control of these impactful plant diseases.

\vspace{1cm}

\begin{objectiveslist}
    \item To develop a theoretical framework for modelling vector-borne
    diseases with non-stationary and non-periodic vector populations.

    \item To investigate the impact of this type of vector dynamics on
    disease spread and severity.

    \item To construct a model for \textit{Xylella fastidiosa} diseases that
    captures the dynamics of the vector population observed in the field.

    \item To validate the developed model using empirical data.

\end{objectiveslist}

\vspace{1cm}

\begin{contributionslist}
    \item We introduce a theoretical framework for modelling vector-borne
    diseases with non-stationary and non-periodic vector populations.

    \item We show that traditional methods to predict the onset of an epidemic
    do not apply in this context and propose new approaches to address this
    challenge.

    \item We develop a model for \textit{Xylella fastidiosa} diseases that
    explicitly incorporates the dynamics of the vector population.

    \item We validate the model using empirical data, demonstrating its
    predictive power and practical utility.

    \item We provide insights into the management and control of
    \textit{Xylella fastidiosa} diseases based on the model predictions.
\end{contributionslist}