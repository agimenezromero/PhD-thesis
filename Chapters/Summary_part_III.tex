\thispagestyle{empty}

\begin{center}
    \textbf{\Large Summary}
\end{center}

This part focuses on the modelling of vector-borne plant diseases, presenting a
detailed exploration of a particular class of non-stationary vector
populations and focusing on the dynamics of \textit{Xylella fastidiosa}
diseases. The need to model vector-borne diseases in which the vector
population follows non-periodic seasonal dynamics, as observed in
\textit{Xylella fastidiosa} diseases, prompted the development of a new
theoretical framework to capture the interplay between vector
populations dynamics and disease spread. This foundational work enabled the
construction of a model for \textit{Xylella fastidiosa} diseases that
explicitly incorporates the complex dynamics of the vector population. Through
this approach, the research addresses current gaps in our understanding of how
non-stationary vector behaviors influence disease spread and severity, offering
new insights into the management and control of these impactful plant diseases.

\vspace{2cm}

\begin{objectiveslist}
    \item To develop a theoretical framework for modelling vector-borne
    diseases with non-stationary and non-periodic vector populations.

    \item To investigate the impact of this type of vector dynamics on
    disease spread and severity.

    \item To construct a model for \textit{Xylella fastidiosa} diseases that
    captures the dynamics of the vector population observed in the field.

    \item To validate the developed model using empirical data.

    \item To explore the implications of the model for the management and
    control of \textit{Xylella fastidiosa} diseases.
\end{objectiveslist}

\vspace{2cm}

\begin{contributionslist}
    \item We introduce a theoretical framework for modelling vector-borne
    diseases with non-stationary and non-periodic vector populations.

    \item We show that traditional methods to predict the onset of an epidemic
    do not apply in this context and propose new approaches to address this
    challenge.

    \item We show how this type of dynamics shape disease spread and severity,
    highlighting the importance of taking into account the dynamics of
    the vector population instead of assuming a constant vector density..

    \item We develop a model for \textit{Xylella fastidiosa} diseases that
    explicitly incorporates the dynamics of the vector population.

    \item We validate the model using empirical data, demonstrating its
    predictive power and practical utility.

    \item We provide insights into the management and control of
    \textit{Xylella fastidiosa} diseases based on the model predictions.
\end{contributionslist}