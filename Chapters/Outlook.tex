
Here we present a general discussion and outlook of the work presented in this
thesis in a broader context: the intersection of Ecology with complex systems
science and artificial intelligence.

Our research during these four years has been focused on interdisciplinary
research. And here I mean really \textbf{interdisciplinary}, not
\textit{multidisciplinary}. Nowadays the term ``interdisciplinary science'' is
used very often, but in many cases it refers to research that is done by a
group of researchers from different disciplines, each one working on their own
part of the problem (i.e. multidisciplinary). In Ecology, it is quite typical
that Ecologists/Biologists (etc) focus on gathering data and performing
relatively easy statistical analyses. After that, they seek collaboration by
passing the data to a group of Physicists/Mathematicians(etc) who will perform
more complex analyses or even develop models. And the same happens in the
opposite direction: there are plenty of Physicists/Mathematicians(etc) who
develop models and then seek collaboration with Ecologists/Biologists(etc) to
validate the models with real data. Both cases are examples of
multidisciplinary research, where the researchers from different disciplines
work on their own part of the problem and then collaborate to put the pieces
together, witout really integrating the different disciplines in the research
process. And don't get me wrong, this is a very good way of doing research, and
it has been very successful in many cases. However, following complex systems
phylosophy, we believe that the whole is more than the sum of its parts.

Most of our research has been really \textbf{interdisciplinary}, or at least
we have tried to make it so.