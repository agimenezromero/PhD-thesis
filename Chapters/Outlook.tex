
Here we present a general discussion and outlook of the work presented in this
thesis in a broader context: interdisciplinary science at the intersection of
Ecology with complex systems and artificial intelligence.

%\section{Interdisciplinary science}

Our research during these four years has been focused on interdisciplinary
(also known as cross-disciplinary) research. And here I mean really
\textbf{interdisciplinary}, not \textit{multidisciplinary}. Nowadays, the term
``interdisciplinary science'' is used very often, but in many cases it refers
to research that is done by a group of researchers from different disciplines,
each one working on their own part of the problem (i.e., multidisciplinary). In
Ecology, it is quite typical that ecologists (biologists, etc.) focus on
gathering data and performing relatively simple statistical analyses. After
that they seek collaboration by passing the data to a group of physicists
(mathematicians, etc.), who will perform more complex analyses or even develop
models. And the same happens in the opposite direction: there are plenty of
physicists who develop models and then seek collaboration with ecologists to
validate the models with real data. Both cases are examples of
multidisciplinary research, where the researchers from different disciplines
work on their own part of the problem and then collaborate to put the pieces
together, without really integrating the different disciplines in the research
process. In any case, this is a very good way of doing research, and it has
been very successful in many cases. However, following complex systems'
philosophy, we believe that the whole is more than the sum of its parts.

Most of our research has been really \textbf{interdisciplinary}, or at least
we have tried to make it so. The biologists and ecologists that we have worked
with have been involved in the whole research process, from the very beginning
to the end. They have been involved in the development of the models, in the
analysis of the results, and in their interpretation. To me, this
is a crucial step in interdisciplinary research, and it is the only way to
really integrate different disciplines in the research process. The trade-off
between \textit{simple enough} mathematical models and \textit{what is indeed
    enough} to capture appropriately the complexity of the system is a
difficult one, and it can only be solved by working together with the experts
in the system that we are studying. Physicists will usually tend to think on
spherical cows in a vacuum, while biologists and ecologists will usually tend
to think on the complexity of the real world, not believing that a simple
model can capture the complexity of the system. So, probably quite often,
physicists will make wrong assumptions, leading to unrealistic conclusions,
while biologists would overcomplicate models, making them intractable and being
unable to generalize the results. The power of interdisciplinary research based
on complex systems' science is that it can bridge this gap, integrating the
knowledge of the experts in the system with the knowledge of the expert
modelers. This can lead to rather general and realistic conclusions that can
be applied to a wide range of systems. In my opinion, this interdisciplinary
approach is the way to address the pressing challenges that we face in the 21st
century, specifically those related with Ecology and Conservation Biology.

%\section{Complex systems}

Complex system science has proven to be a very powerful tool to study the
complexity of the real world. It has been applied to a wide range of systems,
from social systems to ecosystems. The main idea behind complex systems is that
the whole is more than the sum of its parts. This means that the
\textbf{interactions} between the parts of the system are crucial to understand
the system as a whole. Perhaps the best example within this thesis is our
contributions to disease biogeography. As explained in \cref{sec:Disease
    biogeography}, renowned researchers had already mentioned the importance of
interactions to understand the biogeography of diseases \cite{Peterson2008}.
Despite these insights, the assessment of the risk of Pierce's disease had been
hitherto based on the probability of presence of each of the pathosystem
components (i.e., the pathogen and the vector) independently, neglecting their
interactions. Indeed, this led to contrasting conclusions about the future risk
of the disease, as the bacterium is expected to expand its range due to climate
change, while the opposite happens to the vector. A recent study somehow went
into the right direction by assessing the risk of the disease by considering
the overlap of the distributions of the bacterium and the vector with
\textit{enough probability of presence} \cite{YoonLee2023}. Unfortunately, this
was done in a very simplistic way, setting an arbitrary threshold to define
risk areas and, again, neglecting the host-bacterium-plant interactions. It is
clear that a formal framework to integrate the role of interactions into the
modelling of disease biogeography was lacking.

Our mindset based on complex systems' science, together with the collaboration
with biologists and entomologists experts on this pathosystem, has allowed us
to address this issue. We have developed a formal framework to integrate the
role of interactions into the modelling of disease biogeography. Of course,
this has been specifically applied to the case of Pierce's disease, but the
framework is indeed general and can be applied to other diseases and systems.
A project left for the future is to apply this framework to other systems and
formally develop a general theory of disease biogeography based on our
approach.

The universal spatial properties of coral reefs uncovered in
\cref*{ch:coral_reefs} are another example of the intersection of Ecology and
Complex Systems. As commented in \cref{ch:introduction}, power-laws and
fractality are common features of complex systems, usually indicating that the
system is self-organized and adaptive. Our results are a great example on how
very general and regular phenomena can be found in ecological systems
regardless of the specific species that compose the reef. This somehow
resembles the universality of the metabolic theory of ecology, in which the
metabolic rate of organisms scales with body mass with an exponent close to
3/4, regardless of the specific species. This is a very general and regular
pattern that expands across several orders of magnitude of the body mass. In
the case of metabolism, this universality is explained by the fractal-like
structure of the vascular system of organisms and the invariant size of the
capillaries \cite{West1997}. In the case of coral reefs, the universality of
the spatial properties that we found remains to be explained.

%\section{Mathematical modelling}

In this thesis we have developed several mathematical models to study the
complexity of ecological systems, mostly for epidemiology. Following with the
idea of universality, it is noteworthy how, under some conditions, different
diseases can be described by exactly the same mathematical framework that was
developed almost a century ago, the SIR model. A disease caused by a parasite
that affects filter feeders in a marine environment can be mathematically
described in the exact same way that a disease transmitted by an insect vector
that affect plants in a terrestrial environment. From the biological point of
view, these two diseases are completely different: the hosts, the pathogen, the
transmission mechanism, etc. However, from the mathematical point of view,
they can be described by the same equations whenever there exists a timescale
separation between the lifetimes of hosts and parasites/vectors. This is a very
powerful tool, as it allows us to apply the same mathematical framework to a
wide range of systems. This is the power of mathematical modelling in Ecology
from the perspective of complex systems' science.

Of course, these two diseases can be described by exactly the same equations
only under some conditions, namely when the infectious agent (parasite or
infected vector) die fast enough compared to infected hosts. In any case, the
mathematical framework that we use to describe these diseases in the general
case is still the same: compartmental models. This means that, contrary to what
we might expect, the differences between the components of each pathosystem are
not relevant to the mathematical description of the disease, at least at our
scale of description (i.e., the population level). To characterize the dynamics
of the disease at this level we don't need to now the details of the
transmission at a more detailed level, if the pathogen is a bacterium or a
virus, etc. But what we do need to know is how the different components of the
system interact with each other.

Mathematical modelling has been a very powerful tool to study the complexity of
ecological systems. However, it is important to keep in mind that models are
just models, and they are always an approximation of the real world. The
strength of a model is not in its complexity, but in its simplicity. A model
should be as simple as possible, but not simpler. This means that we should
always try to develop the simplest model that can capture the complexity of the
system. This is a very difficult task, and it requires a deep understanding of
the system under study. And that is why interdisciplinary research is
crucial. This is the direction that we have followed in this thesis, and we
believe that this is the way to go in the future. Nevertheless, the
``complexity'' of some challenges is beyond the reach of traditional
mathematical models. Some problems are naturally embedded in high dimensional
spaces, depending on many variables and parameters that are
difficult to measure and quantify. In these cases, the use of artificial
intelligence techniques can be very useful, although it is important to keep in
mind that these techniques are not a panacea.

%\section{Artificial intelligence}

Artificial intelligence (AI) techniques have been used in a wide range of
fields, from computer science to social sciences. In Ecology, the use of
AI techniques is still quite limited, but it is growing. In this thesis we have
developed machine learning models to reconstruct time-series of environmental
variables and map benthic habitats from satellite imagery. However, the use of
these models to obtain ecological insights

In
essence, we have used AI to perform specific tasks, focused on solving a given
technical problem rather than \textbf{understanding} the problem and its
solution. This is a very different approach from the one that we have followed
with mathematical modelling. Nowadays, AI techniques are usually used to solve
specific problems, and they are usually seen as a ``black box'' that gives an
answer to a given problem. This is a very powerful tool, but it is important to
understand what they are meant for.

A clear limitation of AI techniques is that they are really data-hungry. A
typical deep learning model requires thousands of examples to learn a given
task. This is a clear limitation when we are working with ecological systems,
as data are usually scarce and difficult to obtain. Of course, there are ways
to overcome this limitation, such as data augmentation, transfer learning,
etc. To me, these are just workarounds (which can work pretty well in most
cases), as high-quality high-quantity data is a pivotal requirement for AI
techniques to work properly. This is a clear limitation of AI techniques, and
it is important to keep it in mind when we are using them.

At any rate, research in AI is growing very fast, and it is likely that in the
future we will see more and more applications of AI techniques in Ecology. This
is a very exciting field, and it is likely that it will lead to very important
discoveries. However, it is important to keep in mind that AI techniques are
just tools, and they should be used as such. They are not, by any means, a
substitute for a deep understanding of the system that we are studying.
However, this could change in the future with the development of explainable
AI. This is a very active field of research, and it is likely that in the
future we will see AI techniques that are able to explain the reasons behind
their predictions. This would be a game changer, as it would allow us to use AI
techniques to understand the systems that we are studying, not just to solve
specific problems.

%\section{Outlook}
Overall, our research has contributed to the fields of Ecology, complex
systems' science, and artificial intelligence. We have developed different
theoretical and data-driven approaches to advance our understanding of current
challenges related to biodiversity loss. By developing mathematical models of
disease spreading, we have been able to advance our understanding of the Mass
Mortality Event of \textit{Pinna nobilis} and the diseases caused by
\textit{Xylella fastidiosa}. We have build on this knowledge to develop a
formal framework that integrates the role of pathosystem interactions into the
modelling of disease biogeography, which has shown to be crucial to accurately
assess the risk of disease at a global scale. Nevertheless, mathematical models
are not suited to answer all the questions that we face in Ecology, and we have
complemented our research with data-driven approaches. To advance our
understanding of ocean acidification and the impact of climate change on
\textit{Posidonia oceanica} meadows, we have developed machine learning
techniques to reconstruct pH time-series and map benthic habitats from
satellite imagery. Finally, we have investigated the general properties of
coral reefs (which indeed we have found to be universal) by using advanced data
analysis techniques. The general patterns unraveled by our research will help
develop models to better understand the dynamics of coral reefs and their
response to global change.

The future of interdisciplinary research at the intersection of Ecology,
complex systems' science, and artificial intelligence is exceedingly promising.
The pressing challenges of the 21st century demand a profound understanding of
the ecological systems we study, making interdisciplinary research
indispensable. Integrating different disciplines throughout the research
process is crucial for effectively addressing these challenges. Complex
systems' science holds the unique capability to bridge gaps between
disciplines, merging the expertise of system specialists with that of model
experts. This integration fosters the development of comprehensive and
realistic conclusions that can be broadly applied across various systems. As
interdisciplinary research continues to evolve, we can anticipate a growing
number of applications of complex systems' science and artificial intelligence
in Ecology. This thesis aspires to contribute to this dynamic field and aid in
tackling the urgent environmental challenges of our time.

\vfill