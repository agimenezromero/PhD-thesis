
Here we present a general discussion and outlook of the work presented in this
thesis in a broader context: interdisciplinary science at the intersection of
Ecology with complex systems and artificial intelligence.

%\section{Interdisciplinary science}

Our research during these four years has been focused on interdisciplinary
research. And here I mean really \textbf{interdisciplinary}, not
\textit{multidisciplinary}. Nowadays the term ``interdisciplinary science'' is
used very often, but in many cases it refers to research that is done by a
group of researchers from different disciplines, each one working on their own
part of the problem (i.e. multidisciplinary). In Ecology, it is quite typical
that Ecologists (Biologists, etc) focus on gathering data and performing
relatively simple statistical analyses. After that, they seek collaboration by
passing the data to a group of Physicists (Mathematicians, etc) who will
perform more complex analyses or even develop models. And the same happens in
the opposite direction: there are plenty of Physicists who develop models and
then seek collaboration with Ecologists to validate the models with real data.
Both cases are examples of multidisciplinary research, where the researchers
from different disciplines work on their own part of the problem and then
collaborate to put the pieces together, witout really integrating the different
disciplines in the research process. And don't get me wrong, this is a very
good way of doing research, and it has been very successful in many cases.
However, following complex systems phylosophy, we believe that the whole is
more than the sum of its parts.

Most of our research has been really \textbf{interdisciplinary}, or at least
we have tried to make it so. The biologists and ecologists that we have worked
with have been involved in the whole research process, from the very beginning
to the end. They have been involved in the development of the models, in the
analysis of the results, and in the interpretation of the results. To me, this
is a crucial step in interdisciplinary research, and it is the only way to
really integrate different disciplines in the research process. The trade-off
between \textit{simple enough} mathematical models and \textit{what is indeed
    enough} to capture appropriately the complexity of the system is a
difficult one, and it can only be solved by working together with the experts
in the system that we are studying. Physicists will usually tend to think on
spherical cows in a vacuum, while biologists and ecologists will usually tend
to think on the complexity of the real world, not believeing that a simple
model can capture the complexity of the system. So, probably quite often,
physicists will make wrong assumptions, leading to unrealistic conclusions,
while biologists will overcomplicate models, making them untractable and being
unable to generalize the results. The power of interdisciplinary research based
on complex systems science is that it can bridge this gap, integrating the
knowledge of the experts in the system with the knowledge of the experts in the
models. This can lead to rather general and realistic conclusions that can be
applied to a wide range of systems. In my opinion, this interdicsiplinary
approach is the way to adress the pressing challenges that we face in the 21st
century, specifically those related with Ecology and Conservation Biology.

%\section{Complex systems}

Complex system science has proven to be a very powerful tool to study the
complexity of the real world. It has been applied to a wide range of systems,
from social systems to ecosystems. The main idea behind complex systems is that
the whole is more than the sum of its parts. This means that the
\textbf{interactions} between the parts of the system are crucial to understand
the system as a whole. Perhaps the best example within this thesis is our
contributions to disease biogeography. As explained in \cref{sec:Disease
    biogeography}, renowned researchers had already mention the importance of
interactions to understand the biogeography of diseases \cite{Peterson2008}.
Despite these insights, the assessment of the risk of Pierce's disease had been
hitherto based on the probability of presence of each of the pathosystem
components (i.e. the bacterium and the vector) independently, neglecting their
interactions. Indeed, this led to contrasting conclusions about the future risk
of the disease, as the bacterium is expected to expand its range due to climate
change, while the opposite happens to the vector. A recent study somehow went
into the right direction by assessing the risk of the disease by considering
the overlap of the distributions of the bacterium and the vector with
\textit{enough probability of presence}. Unfortunately, this was done in a very
simplistic way, setting an arbitrary thereshold to define risk areas and,
again, neglecting the host-bacterium-plant interactions. It is clear that a
formal framework to integrate the role of interactions into the modelling of
disease biogeography was lacking.

Our mindset based on complex systems science, together with the collaboration
with biologists and entomologists experts on this pathosystem, has allowed us
to address this issue. We have developed a formal framework to integrate the
role of interactions into the modelling of disease biogeography. Of course,
this has been applied specifically for the case of Pierce's disease, but the
framework is indeed general and can be applied to other diseases and systems.
A clear thing to be done in the future is to apply this framework to other
systems and formally develop a general theory of disease biogeography based on
our approach.

Our results on the universal spatial properties of coral reefs are another
example of the intersection of Ecology and Complex Systems. As commented in
\cref{ch:introduction}, power-laws and fractality are common features of
complex systems, usually indicating that the system is self-organized and
adaptive. This is a great example on how very general and regular phenomena can
be found in ecological systems regardless of the specific species that compose
the reef.

%\section{Artificial intelligence}