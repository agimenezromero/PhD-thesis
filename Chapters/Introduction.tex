\setcounter{page}{0}

\section{\label{sec:The global biodiversity crisis} The global biodiversity
  crisis}

%Initial paragraph about ecosystems and biodiversity
Based on our current understanding, life on Earth appears to be unique in the
Universe. Its existence is the result of a long evolutionary process that
started around 3.5 billion years ago \cite{Taylor_1993,Schopf2006},
producing a wide variety of life forms. From the simplest unicellular organisms
to the most complex multicellular forms, the diversity of life is so vast that
it is estimated that there are around 9 million species of living beings in our
planet \cite{Cardinale2012}. This biodiversity is the result of the interaction
among organisms and with their environment, which has led to the development of
complex ecosystems. Ecosystems are the basic units of life on Earth, where
organisms interact with each other and with the environment, forming a
self-regulating system that is capable of maintaining life \cite{Levin2005}.
As Simon Levin eloquently describes:

\begin{displayquote}
  ``\textit{Ecosystems and the biosphere are complex adaptive systems, in which
    pattern
    emerges from, and feeds back to affect, the actions of adaptive individual
    agents, and in which cooperation and multicellularity can develop and
    provide
    the regulation of local environments, and indeed impose regularity at
    higher
    levels. The history of the biosphere is a history of coevolution between
    organisms and their environments, across multiple scales of space, time,
    and
    complexity.}''
\end{displayquote}

% Paragraph on the importance of biodiversity -- ecosystem services
Besides its intrinsic value, biodiversity is essential for the proper
functioning of ecosystems \cite{Gamfeldt2008}, which in turn provide
fundamental services to humankind. These services, known as ecosystem services,
include the production of oxygen, the regulation of the climate, the provision
of food and water, and many others \cite{Daily1997}. Beyond these fundamental
life-supporting benefits, biodiversity also plays critical roles in nutrient
cycling and soil formation, processes integral to the sustainability of our
agricultural systems. The diversity of plant and animal life contributes to
robust ecosystems that can withstand and recover from a variety of disasters,
thereby ensuring ecological resilience. Moreover, biodiversity supports
recreational and tourism industries, which are significant sources of income
for many communities worldwide. The aesthetic and cultural values provided by
diverse ecosystems foster mental and physical health, and contribute to the
cultural heritage of communities, enriching our experience of the world.
Without services ecosystems provide, the stability of environments that support
human life would be greatly diminished, leading to profound economic and social
consequences.

% Paragraph on biodiversity loss, the biodiversity crisis
Unfortunately, the diversity of life on Earth is dramatically diminishing
\cite{Hughes1997,Ceballos2002,Pereira2010}, posing a serious threat to the
stability of ecosystems and the services they provide. Nowadays, wildlife
extinction rates are estimated to be 100 to 1000 times higher than the natural
background rate \cite{Ceballos2015,Pimm2014}, and up to 50\% of higher
taxonomic groups are already critically endangered \cite{Smith2009}. In the
last 50 years, the global population of vertebrates has declined by 69\%, about
50\% of the corals have disappeared due to different causes, mangroves
continue to be deforested at a rate of 0.13\% per year and roughly 10 million
hectares of forests are lost annualy \cite{WWF2022}. Sadly, I could continue
listing more examples of biodiversity loss, but the point is clear: we are
facing a global biodiversity crisis. Indeed, this has led some scientists to
propose that we are entering the sixth mass extinction event in Earth's history
\cite{Barnosky2011}. However, unlike the natural extinction events in Earth’s
past, the current crisis is precipitated by only one species: humans. As
ecosystems falter and species vanish, the intricate web of life that sustains
economies, food security, and our very existence is at risk. If left unchecked,
the repercussions of this biodiversity crisis may lead to ecosystems so
impaired that they no longer fulfill their roles, fundamentally altering the
living conditions on our planet.

% Paragraph on the drivers of biodiversity loss
The main drivers of global biodiversity loss, as identified in the Millennium
Ecosystem Assessment \cite{finlayson2005millennium}, encompass a range of
direct impacts on the natural world. Habitat change, exemplified by the rapid
deforestation in the Amazon Rainforest, results in drastic reductions in
biodiversity by stripping away the complex web of life supported by these
environments \cite{Laurance2012}. Climate change brings about shifts in
temperature and precipitation patterns that are markedly altering habitats, in
particular polar regions, threatening ice-dependent species with extinction
\cite{Post2013}. Indeed, climate change may be a major threat to global
biodiversity in the next 100 years
\cite{Thomas2004,Loarie2009,Pimm2009,Warren2013,Warren2018}, with
predictions for species loss ranging from as low as 0\% to as high as 54\%
\cite{Urban2015}. Invasive species, such as the zebra mussel in North America,
can disrupt ecosystems by outcompeting native species, leading to changes in
the structure of the food webs, affecting the quality and quantity of primary
production, and causing diseases, which probably have been underestimated as an
ecological force \cite{Strayer2010}. Overexploitation of natural
resources, as seen in the overfishing of the world's oceans, can lead to the
collapse of entire ecosystems, as well as the loss of valuable food sources for
human populations \cite{Dayton1995,Coleman2002}. Pollution, particularly from
plastics, infiltrates marine ecosystems globally, endangering marine life
through ingestion and entanglement, and highlighting the pervasive reach of
human waste \cite{Rochman2015}. Finally, wildlife emergent diseases threaten
global biodiversity by potentially producing catastrophic declines in new and
not adapted host populations. If the diseases become endemic, initial
depopulation may be followed by chronic population depression, which could even
lead to local extinction \cite{Daszak2000}. In addition, all this drivers are
interconnected and can have cascading effects on ecosystems \cite{Mora2007}.

% Concluding paragraph
The global biodiversity crisis is a complex and multifaceted problem that
requires urgent action to prevent further loss of species and ecosystems. The
impacts of biodiversity loss are far-reaching, affecting not only the natural
world but also human societies and economies. To address this crisis, we need
to understand the underlying causes of biodiversity loss, predict the effects
of environmental changes on ecosystems, and develop effective strategies for
conservation. This requires a holistic approach that considers the interactions
among species and with the environment, as well as the dynamics of ecosystems
at different scales. In this context, complex systems science provides a
powerful framework to address these challenges.

\section{\label{sec:Complex systems in Ecology} Complex systems in Ecology}

% Paragraph complex systems in Ecology
Complex systems science studies systems composed of many interacting components
whose collective features cannot be understood by simply studying the
individual units in isolation, thus showing ``emergent phenomena''
\cite{Bianconi_2023}. Instead, the behavior of complex systems arises from the
interactions among the components, which can lead to the formation of patterns,
structures, and dynamics that are not present at the individual level.
This emergent behavior is a hallmark of complex systems and illustrates how new
properties and behaviors can arise from relatively simple interactions when
viewed at a larger scale. As a common example, the flocking behavior of birds
is an emergent property of the interactions among individual birds, where each
bird follows simple rules to maintain cohesion with its neighbors, leading to
the formation of complex patterns at the flock level \cite{Vicsek1995}. Complex
systems are also characterized by other features that are ubiquitous in
Ecology, such as non-linear dynamics, feedback loops, self-organization, a lack
of central control, pattern formation, emergence and the presence of multiple
temporal, spatial or organizational scales \cite{Bianconi_2023}. Non-linear
dynamics are extremely common in ecological systems, being exemplified by
predator-prey interactions, where changes in the population of prey can lead to
non-proportional changes in the population of predators, potentially causing
oscillations rather than steady states \cite{Lotka1925}. Self-organization is
observed through the spontaneous formation of spatiotemporal patterns that
arise without any central control, often mediated by scale-dependent feedback
loops, like the formation of vegetation patterns in arid ecosystems
\cite{Rietkerk2008}. Similarly, emergence is seen in phenomena like flocking in
birds or schooling in fish, where group behaviors that cannot be predicted by
studying individual animals emerge from simple rules followed by each member
\cite{Vicsek1995}. Overall, ecological processes operate across multiple
spatial, temporal and organizational scales, from local population interactions
to global biogeochemical cycles, in which each level influences the others.
Indeed, it is not surprising that the principles of complex systems are that
present in ecological systems, as they are composed of many interacting units
whose dynamics are shaped by the interactions among them and with the
environment. For example, ecosystems are composed of many species that
interact with each other and with the physical environment, forming intricate
food webs, nutrient cycles, and energy flows. And the same principles apply to
lower organizational levels such as populations, communities, and
metapopulations, where the interactions among individuals, species, and
habitats give rise to patterns and dynamics that are not always intuitive.

% Paragraph on the application of complex systems in Ecology
\AG{READ PAPER BRUCE T. MILNE}

% Paragraph on the role of models in Ecology
One of the key tools used in the study of complex systems in Ecology is
mathematical and computational models. Models are simplified representations of

\section{\label{sec:Why do we need models?} Why do we need models?}

\section{\label{sec:Thesis structure} Thesis structure}

This thesis is devoted to develop theoretical and data-driven methods to
address timely problems in Ecology and Conservation Biology from the
perspective of Complex Systems. We tackle a variety of current challenges
related to biodiversity loss caused by climate change and emergent diseases,
including expanding vector-borne plant diseases, Mass Mortality Events (MMEs)
in marine ecosystems, ocean acidification, and the decline of important coastal
ecosystems like coral reefs or seagrass meadows. To address these challenges, I
developed a series of theoretical and data-driven methods based on the
principles of Complex Systems, which are presented in the following chapters.
These methods combine theoretical models, data analysis, computational
simulations, and artificial intelligence to study the dynamics of ecosystems,
predict the effects of environmental changes on biodiversity, and design
strategies for conservation. The thesis is organized as follows: