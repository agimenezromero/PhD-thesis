\thispagestyle{empty}

\begin{center}
  \textbf{\Large Summary}
\end{center}

In this part we immerse ourselves into the intricate dynamics of
parasite-induced
marine diseases of immobile hosts, using the mass mortality event of
\textit{Pinna nobilis} as a focal point. We develop and analyse a set of models
aimed at understanding the spread of such diseases in marine ecosystems.
Through the development of the Susceptible-Infected-Recovered-Parasite (SIRP)
model and its spatial counterpart, this work illuminates the mechanisms through
which diseases propagate among marine populations and the potential for
mitigating these outbreaks. Validation against empirical data from the
\textit{Pinna nobilis} mortality event demonstrates the practical utility of
these models in predicting and managing marine diseases. The insights gained
not only contribute to our fundamental understanding of marine epidemiology but
also underscore the critical role of spatial dynamics in shaping disease
outcomes in marine environments.

\vspace{2cm}

\begin{objectiveslist}
  \item To develop a mathematical model for disease
  transmission among sessile marine organisms.

  \item To validate the developed model against real-world data from the
  \textit{Pinna nobilis} mass mortality event.

  \item To extend the model to include spatial dynamics.

  \item Understand how the movement of parasites influences disease spread and
  severity.
\end{objectiveslist}

\vspace{2cm}

\begin{contributionslist}
  \item We introduce the SIRP model, which addresses the
  transmission dynamics of diseases in sessile marine organisms.

  \item  We validate the model with empirical data, establishing a
  strong link between theoretical predictions and observed disease dynamics.

  \item We show the impact of spatial dynamics on disease spread rates and
  epidemic thresholds.
\end{contributionslist}