\thispagestyle{empty}

\begin{center}
  \textbf{\Large Summary}
\end{center}

Since 2016, the Mediterranean Sea has been experiencing a Mass Mortality Event
(MME) of the fan mussel \textit{Pinna nobilis}, caused by the
parasite \textit{Haplosporidium pinnae}. This event has raised concerns about
the conservation of this species, which is listed as endangered by the
International Union for Conservation of Nature (IUCN). The event has also
highlighted the need for a better understanding of the dynamics of marine
diseases, which are often overlooked in the literature. The state-of-the-art
models for disease dynamics in marine ecosystems lag behind those for
terrestrial systems. Compartmental models have been introduced only recently,
lacking a detailed mathematical analysis, and spatial dynamics have been
largely ignored. In this part, we study the dynamics of parasite-induced marine
diseases of immobile hosts, using the MME of \textit{Pinna nobilis} as a focal
point. We develop a compartmental model to describe the transmission dynamics
of diseases in sessile marine organisms. Interestingly, an in-depth
mathematical analysis of the model reveals the conditions under
which these diseases can be modeled as simple SIR-type systems. Validation
against empirical data from the \textit{Pinna nobilis} mortality event
demonstrates the practical utility of the model for future predictions
and management of marine diseases. We then extend the model to include spatial
dynamics and show how the movement of parasites influences disease spread and
severity. The insights gained not only contribute to our fundamental
understanding of marine epidemiology but also underscore the critical role of
spatial dynamics in shaping disease outcomes in marine environments.

\vspace{1cm}

\begin{objectiveslist}
  \item To develop a mathematical model for disease
  transmission among sessile marine organisms.
  \item To validate the developed model against real-world data from the
  \textit{Pinna nobilis} mass mortality event.
  \item To extend the model to include spatial dynamics.
  \item Understand how the movement of parasites influences disease spread and
  severity.
\end{objectiveslist}

% \vspace{1cm}

% \begin{contributionslist}
%   \item We introduce the SIRP model, which addresses the
%   transmission dynamics of diseases in sessile marine organisms.
%   \item  We validate the model with empirical data, establishing a
%   strong link between theoretical predictions and observed disease dynamics.
%   \item We show the impact of spatial dynamics on disease spread rates and
%   epidemic thresholds.
% \end{contributionslist}