\thispagestyle{empty}

\begin{center}
    \textbf{\Large Summary}
\end{center}

In an era where ecological issues are becoming increasingly complex and
globalized, the integration of data-driven methodologies into ecological
research offers unprecedented opportunities for addressing these challenges.
The last part of this thesis explores the application of such methods to
study global ecological problems related to coastal marine ecosystems: the
decline of coral reefs, the acidification of coastal waters, and the loss of
seagrass meadows. Coral reefs are among the most biodiverse ecosystems on
Earth, providing essential services to marine life and coastal communities.
However, they are threatened by a combination of human activities and climate
change, which have led to widespread coral bleaching events and the loss of
coral cover. The acidification of coastal waters is another pressing issue
posing significant risks to marine life, particularly species that rely on
calcium carbonate for their skeletal structures like coral reefs. Seagrass
meadows are also under threat from human activities, such as coastal
development and pollution, which have led to the loss of seagrass habitats and
the decline of associated biodiversity. The understanding of the
spatio-temporal dynamics of these ecosystems together with the monitoring of
their health and factors affecting their resilience is crucial for effective
conservation and management strategies. However, this is often challenging due
to the complexity of these ecosystems and the difficulty of collecting data at
the necessary spatial and temporal scales. By leveraging large existing
datasets, machine learning algorithms, and spatial analysis, here we provide
novel insights into the spatial properties of coral reefs, the reconstruction
of pH time series in coastal waters, and the mapping of seagrass meadows from
satellite imagery. These case studies exemplify the potential of data-driven
approaches to enhance our understanding of ecological dynamics, improve
environmental monitoring, and inform conservation strategies.

\vspace{1cm}

\begin{objectiveslist}
    \item To develop a machine learning model able to reconstruct pH
    time series that present missing data due to sensor failures.
    \item To develop a machine learning model able to map seagrass meadows from
    multispectral satellite imagery.
    \item To investigate the spatial properties of coral reefs at a global
    scale.
\end{objectiveslist}

% \vspace{1cm}

% \begin{contributionslist}
%     \item We obtained universal scaling laws that describe the spatial
%     properties of coral reefs at a global scale.
%     \item We describe 4 macroecological patterns that emerge from the spatial
%     properties of coral reefs: the size-frequency distribution, the
%     perimeter-area relationship, the inter-reef distance distribution and
%     the
%     reefscape fractal dimensions.
%     \item We developed a novel method to reconstruct pH time-series in coastal
%     waters from parallel measurements of temperature, salinity and
%     dissolved
%     oxygen.
%     \item We developed a deep learning model to map seagrass meadows from
%     satellite imagery, which outperforms traditional methods.-
% \end{contributionslist}