
In this thesis we have presented original contributions to Ecology
and Conservation Biology through the lens of complex systems, specifically
focusing on the development of theoretical models and computational
methods to address key challenges in the field. We have contributed to the
advancement of marine epidemiology, vector-borne plant diseases, disease
biogeography, and ecological monitoring, providing new insights into the
dynamics of emergent diseases and the conservation of critical coastal
ecosystems.

Previous to present a general discussion and outlook, we summarize the main
contributions of each part of this thesis. Here we briefly describe the main
results and their implications, and we discuss the remaining questions and
challenges that remain unexplored.

\section{Marine epidemiology}

In \cref{ch:nacras_I} we have developed the
Susceptible-Infected-Recovered-Parasite (SIRP) model, a compartmental model
that describes the transmission dynamics of diseases affecting sessile marine
organisms \cite{GimenezRomero2021}. The model has been applied to
study the Mass Mortality Event (MME) of the fan mussel \textit{Pinna nobilis}
in the Mediterranean Sea. Empirical data on the spread of the disease in
controlled water tanks were used to estimate the parameters of the model and
the basic reproduction number, showing that the model is able to reproduce the
observed disease dynamics. Interestingly, we observed that if the parasites
die fast enough compared to infected hosts, the dynamics of the disease can
be described by a simple SIR model (timescale approximation). This means that,
in this limit, a parasite-produced disease in immobile host populations can be
described by a model in which there is a direct interaction between mobile
hosts. Thus, the parasites ``connect'' the hosts in a way that is formally
equivalent to a direct interaction. In addition, we found an indication that
the transmission of the disease follows an Arrhenius-like temperature
dependence, which allowed us to discern between mass action law and
frequency-dependent transmission.

The natural extension of the SIRP model is the development of a spatially
explicit version. In \cref{ch:nacras_II} we have developed an Individual-Based
Model (IBM) to study the spatial spread of the disease in a realistic marine
environment, showing several interesting results. First, we have shown that the
non-spatial SIRP model describes the spatially extended system when the
dispersal rate of the parasite is fast compared to its death and absorption
rate (by susceptible hosts) \cite{GimenezRomero_2022_RSos}. Second,
counterintuitively, we have shown that the timescale approximation also works
in the spatial model in the low dispersal limit (i.e. when the non-spatial
model is not valid). Third, we have derived an expression for the threshold for
disease invasion in the spatial setting. It consists of the non-spatial
threshold, $R_0$, times a spatial factor that depends on the lifetime and
mobility of the parasite. Finally, we have found that the spreading speed of
the infected population and the time to extinction scale with the dispersal
rate of the parasite.

The development and detailed mathematical analysis of these models has provided
new insights into the dynamics of emergent diseases in marine ecosystems, and
has the potential to inform conservation and management efforts. For instance,
the fact that the dynamics of the disease can be described by a simple SIR
model in the limit of fast parasite dispersal, or that the spatial spread of
the disease can be described by the non-spatial model in the limit of fast
parasite dispersal, is a result that can be generalized to other diseases
affecting sessile marine organisms. This can be crucial when facing emergent
diseases in marine ecosystems, as it provides an easier way to predict the
dynamics of the disease with limited data. Similarly, the closed mathematical
form obtained for the threshold for disease invasion in the spatial model
provides a way to estimate the death rate of the parasite from empirical data
on the spread of the disease if the dispersal rate of the parasite is known.
The results obtained can be compared with the death rate of the parasite in
laboratory conditions. If both measures are inconsistent, this would suggest
that other transmission mechanisms are at play (e.g. an intermediate vector).
In addition, the analysis of the spreading speed of the infected population and
the time to extinction as function of the parasite dispersal rate can be used
to assess the impact of the disease on the host population as function of the
ocean currents. This can be crucial to inform conservation and management
efforts.

Our research on parasite-produced diseases in marine ecosystems has also
provided new questions and challenges that remain unexplored. The MME of the
fan mussel \textit{Pinna nobilis} in the Mediterranean Sea has occurred
in the whole Mediterranean basin, travelling from west to east, in a relatively
short period of time. The mechanisms that have allowed the disease to spread
over such a large area in such a short period of time are still largely
unknown. The SIRP model developed in this thesis can be used to study the
spread of the disease in the whole Mediterranean basin, and to identify the key
factors that have allowed the disease to spread so fast. For this purpose,
metapopulation models that integrate a network of marine currents can be used
to model the spread of the disease in the whole Mediterranean basin. Another
interesting question that remains unexplored is the fact that a PDE model of
the SIRP model does not show the same results as the IBM model with respect to
the threshold for disease invasion. This is an interesting area of research
that has not been explored in this thesis. The lack of stable funding and
the difficulty in obtaining quantitative experimental data about the spread
of the disease in the field, which is crucial to estimate the parameters of the
model, has prevented us from pursuing this line of research. In any case, we
believe that these are interesting and important questions that could be
addressed in future work.

\section{Vector-borne plant diseases}

Moving forward to \cref{ch:xf_PRE}, another key contribution is the
development of a theoretical framework for modeling vector-borne plant diseases
where vector populations follow non-periodic seasonal dynamics, such as growing
or decaying populations towards a stationary state
\cite{GimenezRomero2022_PRE}. This study was motivated by the need to develop
more realistic dynamical models for vector-borne plant diseases, such as those
caused by \textit{Xylella fastidiosa}. Our main finding was that the threshold
for disease invasion in this type of models can not be determined with
traditional methods, such as the Next Generation Matrix (NGM) method. This is
because the initial stage of the pandemic, when the pathogen is introduced
(formally the disease-free state), is not a fixed point of the system, a
necessary condition for these methods to work. Thus, we developed a
new method based on the concept of the basic reproduction number, which
provides a more accurate estimate of the threshold for disease invasion in
these models. In addition, we found that, if the vectors died fast enough
compared to the infected hosts, the dynamics of the disease could be described
by a simple SIR model.

Once the theoretical framework was developed, we were ready to apply it to the
case of \textit{Xylella fastidiosa} diseases. In \cref{ch:xf_phyto}, we adapted
the model to apply it to the case of \textit{Xylella fastidiosa} diseases, in
which the vector population shows non-periodic seasonal dynamics due to its
complex life cycle \cite{GimenezRomero2023}. The model was contrasted with
empirical data from the two main European outbreaks of \textit{Xylella
    fastidiosa} diseases: Almond Leaf Scorch Disease (ALSD) in Mallorca and
Olive Quick Decline Syndrome (OQDS) in Puglia. The model was able to reproduce
the observed dynamics of the disease in both cases and provided valuable
insights into new possible control strategies for the disease. However, we
found that the cross-transmission rates (from infected hosts to vectors and
vice versa) were not identifiable from the data, which only provided
information about the dynamics of the disease in the host population and not in
the vector population. This is an important conclusion of our work: if we want
to study vector-borne plant diseases using more realistic (and complex) models,
we need to have data on the vector population.

Again, the development and detailed mathematical analysis of these models has
provided new insights into the dynamics of vector-borne plant diseases. The NGM
method is widely used to estimate the threshold for disease invasion in
epidemiological models, but we have shown that it is not always applicable
\cite{GimenezRomero2022_PRE}, although a generalization is available in the
case of periodically varying vector populations \cite{Bacaer2006}.
Indeed, it has been already wrongly used in the literature, which can lead to
incorrect conclusions. Our new method provides a correct estimate of the
threshold for disease invasion in models where the vector population is
non-stationary and non-periodic, case in which the NGM method is not
applicable. We have shown again that under certain conditions the dynamics of
the disease can be described by a simple SIR model, which can be crucial to
predict the dynamics of the disease with limited data. Interestingly, these
conditions are exactly analogous to those found in the SIRP model developed in
\cref{ch:nacras_I} for parasite-produced diseases in marine ecosystems of
immobile hosts. This suggests that the dynamics of diseases in immobile host
populations, under some conditions and to some extent, can be described by the
same model regardless of the specific transmission mechanism (in our case a
parasite with no epidemiological states or a vector that can take the
susceptible or infected state). Finally, we have shown that if we want to study
vector-borne plant diseases using more realistic and complex models, we need to
have data on the states of the vector population in addition to hosts.

Our research in this part has also provided new questions and challenges that
remain unexplored. Despite we successfully obtained a method to estimate the
threshold for disease invasion in models where the vector population is
non-stationary and non-periodic, the values we obtain for $R_0$ are not
``realistic'', being usually much higher than 1. We observed that the way in
which the outbreaks were produced was not the same as in the usual epidemic
models. Basically, we did not seem to have a second-order phase transition
between the disease-free and endemic states. This is an interesting area of
research that has not been explored in this thesis.

\section{Disease biogeography}

The last contributions to epidemiology come from the development in
\cref{ch:commsbio} of a mechanistic climate-driven epidemiological model to
assess the risk of establishment of Pierce's disease (PD), caused by
\textit{Xylella fastidiosa} \cite{GimenezRomero2022_CommsBio}. We fitted the
model to empirical data on the distribution of PD in the United States and
showed that the model was able to reproduce the observed distribution of the
disease. We then used the model to assess the risk of PD establishment under
current climate conditions in all viticulture areas worldwide. The most
important result is that, in Europe, the potential distribution of the disease
is currently confined to the Mediterranean basin. In the framework of our
mechanistic risk model this can be understood from by looking at the interplay
between the modified degree-days (MGDDs) and cold degree-days (CDDs)
\cite{GimenezRomero2022_CommsBio}. While low temperatures in winter (high CDD)
prevents the establishment of the disease in continental areas, the mild
temperatures (low CDD) close to coastal areas does not decrease the risk of
establishment of the disease.

The next logical step was to use the model to assess the risk of PD
establishment under future climate conditions. Despite non-coastal
Mediterranean areas in Europe are somehow protected from the establishment of
the disease by now, climate change could change this scenario. In
\cref{ch:xf_climate_change} we have used the latest regional climate change
projections for Europe to assess the risk of PD establishment under future
climate conditions \cite{GimenezRomero2023_PD}.
Our results have shown that there is a critical warming level above which PD
could become established in continental Europe. However, the risk of PD
establishment and its potential impact is not uniform across Europe, with some
regions being more vulnerable than others. Similarly, conclusions are not
uniform across different administrative levels. Some countries can experience
low overall areas at risk, but these areas can be highly productive, thus
translating into a high vineyard area at risk.

These results must depend on the spatial scale of the climate data employed. Up
to that point, we have used climate data at a coarse spatial resolution
(0.1º$\sim 100 \, \textrm{km}^2$), which is the standard in the field.
However, the use of high-resolution climate data can provide more accurate
predictions of the risk of PD establishment taking into account previously
neglected microclimates. In \cref{ch:xf_high_resolution} we have used
high-resolution climate data to assess the risk of PD establishment in
viticulture areas worldwide \cite{GimenezRomero2024}. We expected a small
correction to our previous results, but we found that the potential
distribution of the disease is much larger than previously estimated. The
differences appeared mostly in valleys and rivers, which are precisely the
areas where vineyards are often located. We then analyzed the change in
vineyard locations, showing a significant increase of the vineyard area at risk
worldwide.

Our model advances the field of disease biogeography by providing a mechanistic
understanding of the role of environmental factors in the establishment of
plant diseases, and provides a more reliable methodology than traditional
methods, such as Species Distribution Models (SDMs). Our framework
provides a robust way to map the suitability of the pathosystem components to
the risk of disease establishment by considering the  epidemiological dynamics
of the disease. In addition, our model outcomes naturally provide a measure of
the potential impact of the disease as well as zones with high uncertainty in
the predictions. With this framework, it is straightforward to assess the
impact of different climate change scenarios on the risk of disease
establishment, and to identify the most vulnerable regions. We have provided an
answer to where the disease could become established in Europe under future
climate conditions, which was an open question in the field. In addition, the
analysis of the impact of high resolution climate data on the risk of disease
establishment has shown that previous estimates of the potential distribution
of the disease were underestimated. All our results are crucial to inform
management, prevention and control efforts of plant diseases, such as
\textit{Xylella fastidiosa} diseases.

Further extensions of the model could go in the direction of incorporating
other environmental factors that could affect the establishment of the disease,
such as the presence of alternative hosts or the presence of other vectors.
Similarly, the model could be extended to consider the effect of control
measures, such as vector control or host removal, or include expected mobility
of vectors and infected hosts by plant-trade networks. Finally, the model could
be extended to consider other vector-borne plant diseases than Pierce's
disease.

\section{Ecological monitoring \& analysis}

In the last part of this thesis, we have developed different data-driven
methods for ecological monitoring and analysis. In \cref{ch:ph}, a deep
learning framework based on recurrent neural networks was developed to
reconstruct missing data in ocean pH time series, addressing data gaps in
monitoring ocean acidification and providing reliable pH trend estimates
\cite{Flecha2022}. This has allowed to derive the decadal trend of pH in a
coastal area of the Mediterranean Sea, showing a significant decrease in pH
over the last decades.

This framework can be used to fill gaps in monitoring ocean acidification,
providing reliable estimates of the pH trend that are crucial to properly
assess the impact of ocean acidification on marine ecosystems. The model can
also be used to reconstruct long chunks of missing data in past ocean pH time
series, but one should proceed with caution as the whole context at play could
have changed. Thus, the conclusions derived from the model results should be
taken with care. Future extensions of the model could go in the direction of
incorporating other environmental factors that could affect the pH of the ocean
and developing a generalized approach that could be applied to any region of
the world.

In \cref{ch:posidonia}, another deep learning framework based on convolutional
neural networks was developed to monitor seagrass meadows using satellite
imagery, providing accurate and cost-effective estimates of the extent of
seagrass meadows in the Balearic Islands \cite{GimenezRomero2024_posi}. We have
provided a big step forward in the field. Previous studies were only a proof of
concept for this methodology, often based on a single image or a small dataset
and inappropriate performance measures or model architectures. Furthermore,
these works did not assess the role of image variability and geographic context
in the model performance. In simpler words, these studies were not thought to
be used in practice. Our model, on the contrary, has been trained and validated
on a large dataset of satellite images from the whole Balearic Islands, showing
that the model is able to generalize its learning, at least to the whole
Balearic Islands.

Our results hold significant importance for the conservation of seagrass
meadows, as it provides the basis to monitor these critical ecosystems at a
large scale. The model can be used to obtain the most up-to-date distribution
of \textit{Posidonia oceanica} in the Balearic Islands, and, to some extent, in
the whole Mediterranean Sea. This is crucial to assess the impact of global
change on seagrass meadows, and to inform conservation and management efforts.
Several future extensions of the model could be considered:
monitor seagrass meadows in other regions of the world, include other types of
marine habitats or species, etc. By now, we will retrain our model with other
satellite sources to obtain the distribution of \textit{Posidonia oceanica}
some decades ago. Both results will be crucial to assess the impact of global
change on seagrass meadows.

Finally, in \cref{ch:coral_reefs} we conducted a global analysis of the spatial
properties of tropical coral reefs from mapped remote sensing data, revealing
universal patterns in coral reef size distribution and geometry. We found that
the size distribution of coral reefs follows a power-law distribution, and that
the geometry of coral reefs can be described by a fractal dimension.
Interestingly, we found that the both the exponent of the power-law
distribution and the fractal dimension of coral reefs is independent of the
reefs' location, suggesting that these patterns are universal properties of
coral reefs.

Despite previous work had already shown the presence of power-law size
distributions and fractality in some selected coral reefs, our work is the
first to show that these patterns are indeed universal properties of coral
reefs. It was suggested that fractal dimensions could vary with the
availability of resources, but we have not found significant variations in the
fractal dimension of coral reefs in different regions of the world. Indeed, our
work connects both theoretically and empirically the fractal dimension of coral
reefs with their power-law size distribution, which indicates that these
patterns are related to the same underlying processes, such as the growth and
breakage of coral colonies.

Our results are crucial to challenge existing and future models of coral reef
formation. The power-law size distribution and fractality of coral reefs are
key properties that must be reproduced by any model of coral reef formation.
The universal nature of these patterns suggests that they are related to
fundamental processes that are common to all coral reefs. Indeed, the
macroecological patterns observed for coral reefs are by now unexplained. This
is an interesting area of research that has not been explored in this thesis,
clearly deserving further investigation.