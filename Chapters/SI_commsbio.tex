\section{Inoculation tests on European grapevine
  varieties}\label{app:inoculations}

\textbf{Plants}. Grapevine saplings were annually supplied from a nursery in
mainland Spain (Viveros Villanueva Vides, SL), consisting of one-year-old
rootstocks grafted in winter with dormant grapevine cultivars, and grown in
20-L plastic pots with a standard potting mix. Fifty-seven rootstock-scion
cultivar combinations were used in the inoculation assay (\cref{tableS1}).
Potted plants were randomly distributed in 12-plant rows along an insect-proof
tunnel exposed to air temperature and daily dip-irrigated to field capacity,
fortnightly sprinkled with a slow-release fertiliser and treated with
insecticides and fungicides when needed until the end of the experiment. Two
weeks before the onset of the inoculation assay, leaf samples of all plants
were collected and tested for the presence of Xf through qPCR as described
elsewhere \cite{Moralejo2019}. \\

\noindent\textbf{Isolates and inoculation}. We used for the inoculation
experiment two isolates of Xf. subsp. \textit{fastidiosa} (ST1) recovered from
grapevines: XLY 2055/17 (GenBank WGS: QTJS01) and XYL2177/18 (JAAGVM01)
\cite{Gomila2019,Moralejo2020}. In the third-year assay, we included an isolate
of Xf subsp. \textit{multiplex} ST81 XYL1981/18 (JAAGVE01) to test whether
other strains in Majorca could cause PD as well. Isolates were grown on BYCE
medium at 28ºC for 7-10 days, following EPPO protocols \cite{EPPO2018}.  Cells
were collected by scraping the colonies and suspending them in 1.5 ml Eppendorf
tubes each with 1 ml o1f phosphate-buffered saline (PBS) solution until
obtaining a turbid ($10^8-10^9$ cell/ml) suspension. Plants were mechanically
inoculated by pin-prick inoculation \cite{Almeida2003} with slight
modifications. A 10-$\mu l$ drop of the bacterial suspension was pipetted on
the leaf axil and punctured five times with an entomological needle.
Eight-nine replicates per scion-rootstock combination were inoculated with the
bacterial suspension and four-three plants per cultivar with a drop of PBS as a
control at the end of May. Inoculation was repeated two weeks thereafter by
piercing the next leaf axil above that previously inoculated
\cite{Moralejo2019}.\\

\noindent\textbf{Disease score.} Disease severity was rated by counting the
number of symptomatic leaves eight weeks post-inoculation (WPI) and then
biweekly until the 16th week. A disease index was calculated according to Su et
al. \cite{Su2013}. To determine the basipetal and systemic movement of
Xf$_{\textrm{PD}}$, we counted the number of symptomatic leaves below the point
of inoculation from the same stems or any stem below at 12 WPI. Symptomatic and
asymptomatic plants were tested by qPCR for Xf infection at 12 WPI taking the
petiole of the second and fifth leaf above the point of inoculation. On the
14th week, five leaves per plant of all inoculated plants were used for
Xf$_{\textrm{PD}}$ isolation, as described below. Those plants for which the
qPCR was negative and Xf$_{\textrm{PD}}$ could not be isolated were treated as
not infected.\\

\noindent\textbf{Data analysis}. All statistical analyses on disease scores
were carried out using R. 3.5.2 version software
\cite{R-Development-Core-Team2017}. We used the functions glm and glmer in the
R package lme4 \cite{Bates2015}  for fitting Generalised Linear Models and
Generalised Linear Mixed Models (GLMMs) in the analysis of disease incidence
and severity in the inoculation assays. In all tests, we modelled the response
variable (i) disease incidence with the binomial error (logit‐link function)
and (ii) disease severity with the Poisson error (log‐link function). A
within-subject (repeated measures) factorial design was performed to evaluate
differences in disease severity over time among different cultivar-rootstock
combinations. Cultivars-rootstock and time were treated as fixed factors and
plant subjects as a random effect. Rootstock and time were analysed as fixed
factors and plant subjects as a random effect. Controls were excluded from the
analysis, as lesions did not develop on them. For each cultivar, we calculated
the area under the disease progress curve (AUDPC) from weeks-post-inoculation
and disease index using the package Agricolae. To test whether genotypes within
the ST1-grapevine population vary in virulence, we included a second strain
(XYL2177/18) in the 2019 inoculation experiment. \\

\noindent\textbf{Varietal response to Xf}. In our three-year inoculation tests,
we included a representative number of local and international varieties
(\cref{tableS1}). In total, among 886 inoculated-grapevine plants comprising 36
varieties in 57 unique combinations (scion-rootstock), 86.1\% ($n = 764$) of
them developed typical PD symptoms at 16 WPI. In contrast, none of the
grapevine plants inoculated with the strain XYL 1981/17, ST81 subsp.
\textit{multiplex} presented symptoms. The results of the pathogenicity tests
on European grape varieties are shown in \cref{tableS1}.

Overall, European \textit{V. vinifera} varieties exhibited significant
differences in their susceptibility to Xf$_{\textrm{PD}}$, which could imply
differences in risk of PD establishment at the regional scale (\cref{tableS1}).
When compared between grape major phenotypic groups, red grape varieties were
1.45 times more prone to Xf$_{\textrm{PD}}$ infection than white grape
varieties ($\chi^2= 41.58$, $df=1$, $P=\SI{1.072e-10}{}$), while symptoms were
36.7\% more severe in red grapes than in white grape cultivars
($\chi^2=554.54$, $df=1$, $P=\SI{2.2e-16}{}$).	In addition, we probed whether
Xf$_{\textrm{PD}}$ strains isolated from grapevines in Majorca differ in their
virulence pooled across all grapevine varieties, finding significant
differences in virulence ($\chi^2 = 68.73$, $df = 1$, $P = \SI{2.2e-16}{}$) and
infectiveness ($\chi^2 = 8.07$, $df = 1$, $P =0.0045$) (\cref{figS1}).

Early-season Xf-infections on grapevines are considered to be more likely to
survive the following year than late-season infections
\cite{Feil2001,Lieth2011}. By contrast, varieties developing symptoms, later
on, may affect pathogen acquisition efficiency by vectors and thus decrease the
rate of disease transmission. We found a positive correlation ($F_{1,28} =
    39.58$, $P < 0.001$; $R^2= 0.57$) between the number of symptomatic leaves
formed above the point of inoculation and those formed below (\cref{figS1}).
This acropetal/basipetal ratio of infected leaves is indicative of systemic
movement of the pathogen and of a greater probability that infections on vines
showing a lower number of symptomatic leaves will be more likely eliminated by
winter pruning or by low temperatures \cite{Daugherty2018}. As a result, we
assumed in our model that Xf-infected plants that develop fewer symptomatic
leaves at the end of 16 weeks of incubation will contribute less to the spread
of the disease within vineyards (\cref{figS1}).

\section{Modelling climate suitability for PD}\label{app:S2}
\subsection{Modified Growing Degree Days (MGDD) from Arrhenius
    Equation}\label{app:MGDD} %S2 A

Feil and Purcell estimated Xf growth rate as a function of temperature,
$\sigma(T)$, using Arrhenius’ Law, i.e. $\ln K \sim -1/T$ (see Fig. 3 in
\cite{Feil2001}). The overall dependence on $T$ is nonmonotonic with two
different types of behaviour: i) $k$ grows with $T$ until a maximum value is
attained at $T=\SI{28}{\degree C}=\SI{301.15}{K}$; and ii) $k$ decreases beyond
the maximum. Growth is zero beyond the lowest and highest threshold
temperatures.

The mathematical form of the Arrhenius' Law dependence between the growth
rate $k$ and the absolute temperature $T$ reads as follows,
\begin{equation}
    k = A \exp(-E/T) \ ,
\end{equation}
where $A$ is a pre-exponential factor and $E$ an activation energy in units
of the Boltzmann constant $k_B$. The original use of this equation is for the
rate constant of a chemical reaction that increases monotonically with $T$, and
so $E>0$.  To fit the non-monotonic whole growth behaviour of Xf, we considered
two Arrhenius functions with opposite signs in the activation rate,
\begin{equation}\label{eq: Arrhenius_2}
    k =A_1 \exp(-E_1/T) + A_2 \exp(+E_2/T) \ ,
\end{equation}
where $E_1>0$ and $E_2<0$.\\

Now let us denote by $t$, the temperature in Celsius, $t= T-273.15$.
Importantly, within the bacterial temperature growth range
($10$-$\SI{36}{\degree C}$) in the Arrhenius equation \cref{eq: Arrhenius_2},
$t$ is quite small respect to $b = 273.15$, the absolute (Kelvin) temperature
corresponding to $\SI{0}{\degree C}$ in $T = 273.15 + t$. The two exponents in
\cref{eq: Arrhenius_2} can be approximated now as,
\begin{equation}
    \begin{aligned}
        k=A \exp \left(-\frac{E}{b+t}\right)=A \exp & \left(-\frac{E}{b(1+t /
                b)}\right) \approx A \exp
        \left[-\frac{E}{b}\left(1-\frac{t}{b}\right)\right]=A
        \exp \left(-\frac{E}{b}\right) \exp \left(\frac{E}{b^{2}} t\right)
        \approx
        \\
                                                    & A \exp
        \left[-\frac{E}{b}\right]\left(1+\frac{E}{b^{2}} t\right)=A
        \exp \left(-\frac{E}{b}\right)+A \frac{E}{b^{2}} \exp
        \left(-\frac{E}{b}\right)
        t=B+C t \ ,
    \end{aligned}
    \label{eq:tempapprox}
\end{equation}
where we assume that $t/b = t/273.15 \ll1$ and	$(Et/(b^2) = (Et)/(273.152)
    \ll1$, whereas $B$ and $C$ are constants. In particular, $C>0$ if $E>0$
fits
the region before the maximum in which the growth rate increases, while $C<0$
if $E<0$ fits the region after the maximum where $k$ decreases. The
positive/negative sign stems from the coefficient of the linear term in $t$,
$E/b^2$.

Each exponential in \cref{eq: Arrhenius_2} can be expressed with a simple
straight line, valid in our temperature range. This approach can be extended by
adding more exponential terms in \cref{eq: Arrhenius_2} to further improve the
fit with a multi-linear dependence between Xf$_{\textrm{PD}}$ growth rate and
temperature, obtaining a function proportional to Xf$_{\textrm{PD}}$ growth
rate $F(T)=C\cdot\sigma(T)$ (see \cref{figS2}).

Now, this multi-linear fit to the Xf$_{\textrm{PD}}$ growth rate can be
used to redefine the classical Growing Degree-Days (GDD) metric into the new
Modified Growing Degree Days (MGDD). $GDD$s are computed as the integral of a
particular function of temperature
\begin{equation}
    GDD(t)=\int_{t_0}^tf(T(t))\dif t
\end{equation}
where $f(T)$ is defined as
\begin{equation}
    f(T(t))=\left\{\begin{array}{ccc}
        T(t) - T_{\textrm{base}} & \textrm{if} & T\geq T_{\textrm{base}} \\
        0                        & \textrm{if} & T < T_{\textrm{base}}
    \end{array} \right.
\end{equation}

Considering different slopes relating Xf$_{\textrm{PD}}$ growth rate and
temperature at different temperature intervals, as shown in \cref{figS2}, we
modified this particular function to now account for Xf$_{\textrm{PD}}$ growth,
\begin{equation}\label{eq:MGDD_t}
    MGDD(t)=\int_{t_0}^tF(T(t))\dif t=C\cdot\int_{t_0}^t\sigma(T(t))\dif t
\end{equation}

We wish to stress that the use of a multilinear form to represent MGDDs
\cref{figS2} stems from the fundamental temperature dependence of the kinetics
of bacterial growth as described by the Arrhenius equation, and is not an
arbitrary simplified representation. Moreover, the MGDD function is fitted
using the whole set of data published in \cite{Feil2001}, and is not simply
based on the knowledge of the cardinal temperatures, as customarily done when
writing smooth interpolating functions with the sole input of the cardinal
temperatures (see, e.g., \cite{Yan1999}).

\subsection{Relation between MGDD and within-plant bacterial
    population}\label{app:MGDD_growth} %S2 B

The usual growth cycle of bacteria consists of several phases (lag,
exponential, stationary and death phase), being of most interest to
environmental microbiologists the interval between the lag and the onset of the
stationary phase  \cite{MAIER200937}. During the exponential phase, the rate of
increase of cells is proportional to the number of cells present at any
particular time. Thus, the evolution of the bacterial population, $N$, over
time is given by the following differential equation,
\begin{equation}
    \der{N}{t}=\sigma N \Longrightarrow N(t)=N_0\cdot\exp(\sigma t) \ ,
\end{equation}
where $\sigma$ is the specific growth rate constant.

As shown in the previous section, the growth rate of Xf has specific
temperature dependence, $\sigma(T)$. In our study, temperature varies over
time, so we can write the growth rate as a time-dependent quantity,
$\sigma(T(t))$. With this, the evolution of the bacterial population will be
given by
\begin{equation}
    N(t)=N_0\exp(\int_{t_0}^{t_f}\sigma(T(t)) \dif t) \ .
\end{equation}
Recalling \cref{eq:MGDD_t} we can write the previous equation as
\begin{equation}
    N(t)=\frac{N_0}{C}\cdot\exp(MGDD(t))=C'\cdot N_0\exp(MGDD(t))
\end{equation}
Indeed, the same can be done considering the logistic differential equation
(that includes the stationary phase),
\begin{equation}
    \der{N}{t}=\sigma(T(t))\cdot N\cdot\parentesi{1-\frac{N}{K}}
\end{equation}
whose solution is
\begin{equation}
    N(t)=\frac{K}{1+C\exp(-\int_{t_0}^{t_f}\sigma(T(t)))}
\end{equation}
and using \cref{eq:MGDD_t} it can be rewritten as
\begin{equation}\label{eq:MGDD_bacteria_relation}
    N(t)=\frac{K}{1+C'\exp(-MGDD(t))}
\end{equation}
and thereby the bacterial population after a given time $t$ is related to
the $MGDD$ by \cref{eq:MGDD_bacteria_relation}.\\

\textbf{Note:} We are assuming a correspondence between \textit{in vitro}
and \textit{in planta} growth rates of Xf.

\subsection{Epidemiological and theoretical basis}\label{app:SIR}
%S2 C

A standard SIR model was considered as a basis to assess the risk of PD
outbreaks worldwide (see \cref{app:SIR_justifications} for an analytical
derivation of the relation between a vector-borne disease model and a standard
SIR model). The model is represented by the following three equations,
\begin{equation}
    \begin{array}{l}
        \dot{S}=-\beta SI/N           \\
        \dot{I}=\beta SI/N - \gamma I \\
        \dot{R}=\gamma I \ ,
    \end{array}
    \label{eq:SIRmodel}
\end{equation}
where $S$ is the susceptible host population, $I$ is the infected
population, $R$ is the dead population and the total population $N$ is
conserved, $S+I+R=N$, as hosts die only when they contract the disease.
The transmission of the disease from infected hosts to susceptible ones is
mediated by the \textit{transmission rate} $\beta$ while the death of infected
individuals is regulated by the \textit{mortality rate} $\gamma$.

Analysing the non-trivial fixed point, $\vec{x}=(N, 0, 0)$, it can be
proved the existence of an epidemic threshold. As $S$ is a monotonically
decreasing function, which implies $S(t)<S_0$, one can write the following
relation,
\begin{equation}\label{eq: threshold_t}
    \der{I}{t}=I\parentesi{\beta S/N-\gamma}\leq I\parentesi{\beta
        N/N-\gamma}=\gamma I\parentesi{\beta/\gamma-1}=\gamma
    I\parentesi{R_0-1} \ ,
\end{equation}
where $R_0= \beta/\gamma$. Thus, $R_0<1$ implies $dI/dt<0 \ \forall t$ and
$I_0>I(t)$ as $t\to\infty$, basically meaning that the epidemic dies out, while
for $R_0>1$, $I(t)$ grows initially until $S(t_c)=\gamma/\beta$, at which
$\dot{I}(t_c)=0$, and the epidemic starts waning out.
$R_0$ corresponds to the so-called \textit{basic reproduction number} and
measures the number of secondary infections given by a primary infection in a
fully susceptible population.

We wish to model the risk of PD establishment in a susceptible (healthy)
population. For this, we characterised the maximum growth rate of the epidemic,
when $S(t)\sim S(0)$. Thus, the growth is well approximated under these
conditions with
the (linearised) differential equation,
\begin{equation}
    dI/dt=\beta SI-\gamma I\approx \gamma I(\beta N/\gamma - 1)=\gamma
    I(R_0-1) \ .
\end{equation}
where we have assumed the initial conditions,
$S_0\approx N$, $I(0)\approx0$ and $R(0)=0$. This linear differential
equation can be integrated exactly,
\begin{equation}\label{eq: infect_proc}
    I(t)=I(0)\exp(\gamma(R_0-1)t) \ .
\end{equation}
As explained in the main text, to account for the effect of temperature in
the epidemic process we modify the previous expression as follows
\begin{equation}\label{eq: final_eq}
    I(t)=I(0)\exp(\gamma(R_0-1)t)\cdot \mathcal{F}\parentesi{MGDD(t)}\cdot
    \mathcal{G}\parentesi{(CDD(t)}= I(0)\exp(\gamma(R_0-1)t)\cdot\Pi(t) \ ,
\end{equation}
where $\Pi(t)=\mathcal{F}(MGDD(t))\cdot \mathcal{G}(CDD(t))$ is the
cumulative probability of chronic infection that depends on temperature.

\subsection{Determination of $R_0$ for Europe}\label{app:R0_Europe}
%S2 D

Unlike the validation of our model based on the distribution of PD in the
USA, there are no spatiotemporal data on PD outbreaks available in Europe to
estimate $R_0$. One way to solve this problem is to use data on the incidence
of almond leaf scorch disease in Majorca to fit a SIR model and obtain an
approximation of $R_0$. The initial date of introduction and progression of the
almond leaf scorch epidemics is well characterised and both diseases are
transmitted by \textit{P. spumarius} \cite{Moralejo2019,Moralejo2020}. Using
$\gamma=1/14 \textrm{years}^{-1}$ as the mortality rate \cite{Moralejo2020},
the best fit was provided by $\beta_{\textrm{opt}}=0.8$, giving rise to
$R_0=11.2$, which is in good agreement with the order of magnitude of $R_0=8$
in the United States (\cref{fig:sup_ROC}). To find a proper scenario for PD in
Europe, we considered a constant transmission rate
$\beta_{\textrm{opt}}$\footnote{As the vector that transmits both ALSD and PD
    is the same (\textit{Philaenus spumarius}) we considered that the
    transmission
    rate should not vary much.} and applied an average mortality rate
$\gamma\sim
    1/5 \textrm{years}^{-1}$ of PD infected vines \cite{Purcell2013}, which
gives
rise to $R_0=4$. Finally, we used the information on the climate suitability
for the vector in Majorca ($\approx0.8$ on average) to determine a baseline
scenario for Europe, $R_0=4/0.8=5$. Thus we can argue that $R_0=5$ is a good
proxy for modelling the establishment of PD in Europe. The use of $R_0=5$ is
furthermore corroborated by the reasonability of the predictions obtained.

\subsection{Simulation details}\label{app:risk_index}
%S2 E

To assess the risk of \textit{Xf} establishment in vineyards, we performed
spatiotemporal simulations for the world's largest wine-growing areas. The cell
size of the abstract grid was determined by the resolution of the data
collected from ERA5-Land, $0.1$ \textdegree $\cross 0.1$ \textdegree , so the
spatial resolution is approximately $\SI{9}{km}$ in the latitudes of the
Mediterranean basin. A small initial infected-plant population was introduced
annually into each cell assuming that if the conditions are or become
favourable the disease will propagate locally. We chose $I_i(0)=1$ to re-scale
the results to any initial population size, and implemented \cref{eq: final_eq}
in each cell. Simulation time was discretised in years and computed in two
steps incorporating summer ($F(MGDD)$) and winter ($F'(CDD)$) periods. To
implement \cref{eq: final_eq} we took into account that the $MGDD$ and $CDD$
differ at each time step; thereby it required to convert \cref{eq: infect_proc}
into a mathematical map. The equation can be expressed as,
\begin{equation}
    I(t)=I(0)\exp(\gamma(R_0-1)t)=I(0)\claudator{\exp(\gamma(R_0-1))}^t \ ,
\end{equation}
where $t$ is the discrete-time in years,
so that
\begin{equation}
    I(t-1)=I(0)\claudator{\exp(\gamma(R_0-1))}^{t-1} \ ,
\end{equation}
and, thus,
\begin{equation}
    I(t)=I(t-1)\exp(\gamma(R_0-1)) \ .
\end{equation}
The discretized form of \cref{eq: final_eq} is then
\begin{equation}
    I(t_i)=I(t_{i-1})\exp(\gamma(R_0-1))\cdot F(MGDD(t_i))\cdot
    F'(CDD(t_i)) \ .
    \label{eq:imapevol}
\end{equation}

A risk index was created to represent the relative velocity of PD local
exponential propagation,
\begin{equation}
    r(\tau)=\textrm{max}\left\{
    \frac{\log(I(\tau)/I(0))}{\gamma(R_0-1)\tau}, -1
    \right\} \ ,
    \label{eq:riskindex}
\end{equation}
where $\tau$ is the simulated time, $R_0$ is the basic reproduction number
and $I(0)$ the initial condition (initial number of infected plants). The index
ranges from -1 to 1 as the maximum risk value always occurs under optimal
climatic conditions ($F(MGDD)=F(CDD)=1)$ and thus
$I(\tau)=I(0)\exp((R_0-1)\tau)$. The minimum risk was intentionally cut off at
-1 to use a symmetric scale, as otherwise, the logarithmic scale is unbounded.

The numerator of the risk index defined in \cref{eq:riskindex} is formally
similar to the definition of Lyapunov exponents (LEs), which characterise
predictability in chaotic systems (the denominator normalises this quantity to
its maximum value). This is not surprising because both the risk of
\cref{eq:riskindex} and the growth of perturbations in chaotic systems
correspond to an exponential process.
Following this analogy, we would expect a growing exponential process in
the risk of the establishment if $r>0$, while a decreasing exponential that
goes to $0$ would denote no risk if $r<0$.
However, Lyapunov exponents are (normally) calculated for autonomous (i.e.
unforced, and so \textit{steady}) dynamical systems, while \cref{eq:imapevol}
has $2$ forcing terms (i.e. is non-autonomous).
The result is a non-exponential behaviour found when $|r|$ is small. So
beyond the expected regions with growing exponential and decreasing exponential
behaviour, we find a transition zone, where the system is oscillatory and not
exponential, as neither growth in more auspicious years for Xf$_{\textrm{PD}}$
or decrease in less auspicious ones prevails, and neither of the growing or
decreasing pure exponential behaviours manifests.

We define the borderlines of this transition region by $I(\tau)\leq10 \cdot
    I(0)$ in the southern boundary and $I(\tau)\geq 0.05 \cdot I(0)$ in the
northern one when $\tau=40$ years. Basically, for the upper boundary, we assume
that if an initial infection is multiplied by 10 after 40 years, then the
exponential growth would be unstoppable. Conversely, if an initial introduction
of infected individuals decays more than 95\% of its original value after 40
years, we then assume that the exponential decay would continue and clearly PD
cannot be established. Since $\tau, \gamma$ are fixed, the limits of the
transition zones depend on $R_0 $ and it is given by the risk index instead of
the number of infected plants as follows,
\begin{equation}
    \begin{array}{ll}
        \displaystyle \textrm{Upper limit:} & \displaystyle

        r_{\textrm{trans}}^{\textrm{max}}=\frac{\log(10)}{\gamma\parentesi{R_0-1}\tau}
        \\\\
        \displaystyle \textrm{Lower limit:} & \displaystyle

        r_{\textrm{trans}}^{\textrm{min}}=\frac{\log(0.05)}{\gamma\parentesi{R_0-1}\tau}
    \end{array} \ .
\end{equation}

For instance, with $\gamma=\SI{0.2}{years^{-1}}$, $\tau=\SI{39}{years}$ and
$R_0=5$ (values used for Europe) the transition zones are delimited by
$-0.09<r(\tau)<0.075$. So, the model outputs can be associated with the
following behaviours:
\begin{enumerate}
    \item Epidemic-risk zones: $r(\tau)>r_{\textrm{trans}}^{\textrm{max}}$.
          The risk index $r_j(\tau)$ is ranked as high ($r_j(\tau) >$ 0.9),
          moderate
          (0.9-0.66), low (0.66-0.33) and very low
          (0.33-$r_{\textrm{trans}}^{\textrm{max}}$).
    \item Transition-risk zones:

          $r_{\textrm{trans}}^{\textrm{min}}<r(\tau)<r_{\textrm{trans}}^{\textrm{max}}$.
          In this zone the incidence, $I(t)$, predicted by the model does not
          grow
          clearly, but neither it does disappear, and incidence oscillates.
          This region
          is expected to be very sensitive to changes induced by climate
          change, and
          transit to epidemic-risk zones with low growth rates.
    \item Non-risk zone: $r(\tau)<r_{\textrm{trans}}^{\textrm{min}}$.
          Incidence decrease exponentially due to the combined effect of the
          $MGDD$ and
          $CDD$, or to the (low) vector abundance in the case of predictions
          for Europe.
          Cells in this region with $r_j$ not far from $-0.1$ could become
          transitional
          due to the effect of climate change.
\end{enumerate}

\subsection{Vector distribution influence}\label{app:vector_influence}
%S2 F

Information on the climatic suitability of the vector \textit{P. spumarius}
\cite{Godefroid2021} was used to modulate the value of the basic reproduction
number. We assumed a linear dependence of $\beta$, the transmission rate, with
the vector climatic suitability resulting in each of the model cells,
\begin{equation}
    R_0(x)=\frac{\beta v(x)}{\gamma}=R_0\cdot v(x) \ ,
\end{equation}
where $x$ illustrates the space dependence.

In \cref{app:SIR_justifications} we show an analytical derivation of the
linear dependence between $R_0$ and the vector population (i.e. the number of
vectors). Then, assuming that climatic suitability (i.e. probability of
presence) is directly related to the number of vectors we obtain the linear
scaling between $R_0$ and climatic suitability for vectors.

\section{Future risk extrapolation}\label{app:future}
%S3

To project PD risk in a climate change scenario, historical \textit{CDD}
and \textit{MGDD} data were calculated to generate annual time series for each
location recorded in the data set. To obtain the time trend of the variables in
each pixel, a linear model was fitted using Sklearn's LinearRegression module
in Python \cite{scikit-learn}. The interannual climatic variability was also
included as a Gaussian noise distribution by calculating the mean and
fluctuations of the variance around the trend of the $MGDD$ and $CDD$ metrics
for any record in the data set. We show in \cref{figS12} the determination of
the trend of the metrics $MGDD$ and $CDD$ for Lecce and Bordeaux. \cref{figS13}
shows three realisations to extrapolate the $MGDD$ and $CDD$ metrics for
Bordeaux after applying Gaussian noise to the trend. This risk extrapolation to
2050 implies a linear extrapolation of past MGDD and CDD tendencies. Note that
because $MGDD$ and $CDD$ functions are nonlinear this is just a rough
approximation to the future risk, as non-linearities could play a major role in
a climate change scenario.

% S4
\section{Mathematical justifications}\label{app:SIR_justifications}

We show how a linear scaling between the vector population and the basic
reproduction number can be obtained from a vector-borne disease model.
Moreover, a SIR model can be derived from the same vector-borne disease model
(under some assumptions).

In a model defined according to the following processes,
\begin{equation}\label{eq:scheme_infection}
    %\varnothing \stackrel{\delta}{\rightarrow} S_V \quad 
    S_H+I_V \stackrel{\beta}{\rightarrow} I_H + I_V \quad I_H
    \stackrel{\gamma}{\rightarrow} R_H \quad S_V+I_H
    \stackrel{\alpha}{\rightarrow}
    I_V+I_H \quad S_V \stackrel{\mu}{\rightarrow} \varnothing \quad I_V
    \stackrel{\mu}{\rightarrow} \varnothing
    \ ,
\end{equation}
where the birth of new susceptible vectors is described as a source term,
the host-vector compartmental model can be written as,
\begin{equation}\label{eq:SIR_v}
    \begin{aligned}
        \dot{S}_H & =-\beta S_H I_v / N_H                     \\
        \dot{I}_H & =\beta S_H I_v / N_H - \gamma I_H         \\
        \dot{R}_H & =\gamma I_H                               \\
        \dot{S}_v & = \delta C-\alpha S_v I_H / N_H - \mu S_v \\
        \dot{I}_v & =\alpha S_v I_H / N_H - \mu I_v \ ,
    \end{aligned}
\end{equation}
when a standard incidence \cite{MartchevaBook} is considered.

The model describes the infection of susceptible hosts ($S_H$) at a rate
$\beta$ through their interaction with infected vectors ($I_v$), while
susceptible vectors ($S_v$) are infected at a rate $\alpha$ through their
interaction with infected hosts ($I_H$). Infected hosts exit the infected
compartment at a rate $\gamma$, while infected vectors stay infected for the
rest of their life since they are not affected by the pathogen. The model
assumes that vectors die naturally (or disappear from the population by some
mechanism) at a rate $\mu$ and are born (appear) at a constant rate $\delta$,
being susceptible. The constant term $C$ sets the scale of the stationary value
of the vector population.

\subsection{Linear scaling of $R_0$ with vector population}

The standard methods of calculation of $R_0$ are based on the linear
stability analysis of the disease-free equilibrium, either directly, through
the linear analysis of the fixed point that yields the stability condition from
which $R_0$ can be obtained, or using the Next Generation Method (NGM)
\cite{Diekmann2010} that provides directly $R_0$ by solving a suitable linear
problem. The disease-free equilibrium of the model (the fixed point) is given
by $I_H=I_v=0$ yielding $\dot{S_v}=0\Longrightarrow S_v=\delta C/\mu=N_v^*$,
where $N_v^*$ is the stationary value of the vector population.\\

As shown in \cite{vectoresAG}, both methods yield the following relation
for the basic reproduction number,
\begin{equation}
    R_0 = \frac{\beta \alpha}{\gamma
        \mu}\frac{C}{N_H}\frac{\delta}{\mu}\frac{S_H(0)}{N_H}=\frac{\beta
        \alpha}{\gamma \mu}\frac{N_v^*}{N_H}\frac{S_H(0)}{N_H}\ ,
\end{equation}

in which the basic reproduction number scales linearly with the vector
population.

\subsection{Reduction to a SIR model}

In a time-scale where the vector population changes faster than the host
population (a good approximation for Xf$_{\textrm{PD}}$-related diseases), the
former will almost instantaneously reach the stationary value. Thus, if
$1/\mu\ll1/\gamma$, or equivalently if $\mu\gg\gamma$, we can rewrite the time
derivative of the vector infected population as
\begin{equation}
    \epsilon\dot{I}_v=\frac{\alpha}{\mu}S_v\frac{I_H}{N_H} - I_v \ ,
\end{equation}
with $\epsilon=1/\mu$ being a small parameter. Then, $\dot{I_v}$ can be
neglected and the infected vector population can be obtained from the
relationship,
\begin{equation}\label{eq:Iv_timescale_approx}
    I_v\approx\frac{\alpha}{\mu}\frac{S_v I_H}{N_H} \ .
\end{equation}

Furthermore, if $\lambda N_H \gg I_H$ (which is indeed plausible in this
limit) the model can be written as a SIR model with constant coefficients,

\begin{equation}\label{eq:SIR}
    \begin{aligned}
        \dot{S}_H & =-\beta_{eff}\frac{S_HI_H}{N_H}            \\
        \dot{I}_H & =\beta_{eff}\frac{S_HI_H}{N_H}- \gamma I_H \\
        \dot{R}_H & =\gamma I_H \ ,
    \end{aligned}
\end{equation}
where $\displaystyle\beta_{eff}=\frac{\beta'}{\lambda}=\frac{\beta\alpha
        N_v^*}{\mu N_H}$.\\

Note that in the SIR model reduction, the effective $\beta_{eff}$
coefficient depends linearly on the vector population $N_v^*$.

\newpage

\section{Figures}

\begin{figure}[H]
    \centering
    \includegraphics[width=1\textwidth]{Figures/Fig S1.pdf}
    \caption{\textbf{Factors influencing \textit {Xf-Philaenus
                spumarius-Vitis vinifera} pathosystem.}\textbf{(a)} Virulence
        differences
        between \textit{Xf} subsp. \textit{fastidiosa} isolates on grapevines.
        Bars
        represent the mean number of symptomatic infected leaves four months
        after
        inoculating. Both isolates XYL2055/17 ($n$ = 316 inoculated plants) and
        XYL2177/18 ($n$ = 260) were collected from vineyards on Majorca. Scores
        were
        pooled among the 21 varieties inoculated;\textbf{(b)}conceptual graph
        of the
        population dynamics of \textit{P. spumarius} on vineyards in Majorca
        and the
        effect on winter curing. Blue: density function of \textit{P.
            spumarius}; red
        line: proportion of \textit{P. spumarius} carrying Xf; blue line:
        proportion of
        plants recovering according to the time they are infected;\textbf{(c)}
        bimodal
        density function of the number of symptomatic leaves. The blue dash
        line marks
        5 symptomatic leaves; and \textbf{(d)}correlation between the upward
        and
        downward movement of Xf$_{\textrm{PD}}$ within the canes from the
        inoculation
        point. Each point depicts the mean distance travelled in both
        directions by the
        bacteria.
        \label{figS1}} %S1
\end{figure}

\begin{figure}[H]
    \centering
    \includegraphics[width=1\textwidth]{Figures/Experimental setup.jpg}
    \caption{\textbf{Experimental setup.} Greenhouse facilities and general
        view of its interior and the arrangement of the vine plants. The
        metallic
        structure is covered with an anti-thrips mesh.}
    \label{fig:experimental_setup} %S2
\end{figure}

\begin{figure}[H]
    \centering
    \includegraphics[width=0.6\textwidth]{Figures/Climatic_layer_1.png}
    \caption{\textbf{Relationship between MGDD and temperature.}
        Contribution to the $MGDD$ resulting from the fitting to the data in
        (1). The
        original Arrhenius plot, $\log{k} \ vs. \ 1/T$ in kelvin was converted
        to
        a linear dependence in Celsius temperature $t$ (cf.
        \cref{eq:tempapprox})}.
    \label{figS2} %S3
\end{figure}

\begin{figure}[H]
    \centering
    \includegraphics[width=0.6\textwidth]{Figures/CDD_risk_evol_USA.png}
    \caption{\textbf{Trends in the risk-epidemic zones during the 1981-2019
            period (A) and the areas encompassed below the $CDD < 314$ line (B)
            comprising
            land areas between $103^o$W and $70^o$W of the United States}}.
    \label{fig:sup_CDD_evol} %S4
\end{figure}

\begin{figure}[H]
    \centering
    \includegraphics[width=0.6\textwidth]{Figures/ROC_curve.png}
    \caption{\textbf{ROC curve illustrating the model validation procedure
            with spatiotemporal data from PD distribution in the US.} TPR is
        the
        \textit{true positive rate} and FPR the \textit{false positive rate}.
        The model
        accuracy reaches its optimum in $R_0\approx8$ by maximising the true
        positive
        rate and minimising the false positive rate. The spatiotemporal PD
        distribution
        in the US was obtained from data collected from publications between
        2001 and
        2015.}
    \label{fig:sup_ROC} %S5
\end{figure}

\begin{figure}[H]
    \centering
    \includegraphics[width=0.6\textwidth]{Figures/R0_xylella.png}
    \caption{\textbf{Fitting a $SIR$ model to the progress of the almond
            leaf scorch disease in the Balearic Islands from 1993 onward}. The
        best match
        was obtained with $R_0= 11.2$. Points represent an estimate of the
        proportion
        of infected trees (incidence) from dendrochronological analysis and
        detection
        of Xf DNA in growth rings by qPCR. The incidence of ALSD in 2012 and
        2017 was
        independently validated by field and Google Map Street View image
        observations.}.
    \label{fig:sup_R0_Europe} %S6
\end{figure}

\begin{figure}[H]
    \centering
    \includegraphics[width=0.65\textwidth]{Figures/validation.png}
    \caption{\textbf{Model validation for an $R_0=8$ scenario with
            presence/absence data (black/white stars) of PD in the United
            States.} Panel
        (A) corresponds to data from California in 2015 while the other panels
        show
        data from 2002, 2005, 2006 and 2001 (respectively) in the east of the
        United
        States. The last panel clarifies the validation zones previously
        mentioned.}
    \label{fig:sup_validation} %S7
\end{figure}

\begin{figure}[H]
    \centering
    \includegraphics[width=0.8\textwidth]{Figures/vector_layer.png}
    \caption{\textbf{Average climatic suitability for \textit{Philaenus
                spumarius} in Europe.} The map shows the climatic suitability
        of the vector
        estimated from a generalized additive model of insect distribution and
        the
        correlation of two bioclimatic descriptors, a climatic humidity index
        for the
        period of 8 coldest months of the year and the average maximum
        temperature in
        spring.}
    \label{fig:sup_vector} %S8
\end{figure}

\begin{figure}[H]
    \centering

    \includegraphics[width=0.7\textwidth]{Figures/Mean_MGDD_CDD_sites.png}
    \caption{\textbf{Trends in $MGDD$ (A) and $CDD$ (B) values and
            oscillations during 1981-2019 for seven wine regions with different
            climates
            from Europe and the US.} $MGDD$s show a slight upward trend and
        lesser
        oscillations than the $CDD$s.}
    \label{fig:sup_climatic_oscilations} %S9
\end{figure}

\begin{figure}[H]
    \centering
    \includegraphics[width=0.7\textwidth]{Figures/MGDD_CDD_trends.png}
    \caption{\textbf{Determination of $MGDD$ and $CDD$ metric trends and
            future projections for two different European regions.} The $MGDD$
        (A) and
        $CDD$ (B) trends show steeper slopes in the temperate climate of
        Bordeaux than
        in the Mediterranean climate of Lecce.}
    \label{figS12} %S10
\end{figure}

\begin{figure}[H]
    \centering

    \includegraphics[width=0.8\textwidth]{Figures/MGDD_CDD_scenarios.png}
    \caption{\textbf{Interannual climatic variability extrapolations of
            $MGDD$	(A) and $CDD$ (B) for Bordeaux.} A linear model was
        fitted using
        Sklearn's LinearRegression module in Python and the interannual
        climatic
        variability was included as a Gaussian noise distribution by
        calculating the
        mean and fluctuations of the variance around $MGDD$ and $CDD$ trends.}
    \label{figS13} %S11
\end{figure}

\begin{figure}[H]
    \centering
    \includegraphics[width=0.9\textwidth]{Figures/Risk_risk_beta.png}
    \caption{Risk index computed with (A) MGDD from the Arrhenius-based fit and
        (B) MGDD from the beta function fit.}
    \label{fig:R2} %S12
\end{figure}

\begin{figure}[H]
    \centering

    \includegraphics[width=\textwidth]{Figures/Diff_risk_beta_minus_risk.png}
    \caption{Difference in risk index when computed using MGDD calculated from
        the Arrhenius-based fit or the beta function fit.}
    \label{fig:R3} %S13
\end{figure}

\section{Tables}

All tables can be found at the Online Supplementary Information of:

\begin{center}
    \fullcite{GimenezRomero2022_CommsBio}
\end{center}

%References
%\nocite{*}
%\bibliography{refs_SI.bib}
