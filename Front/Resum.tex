\pagebreak
\thispagestyle{empty}

\begin{center}
    \textbf{\Large Resum}
\end{center}

La vida a la Terra ha evolucionat al llarg de milions d'anys,
produint una rica diversitat d'espècies i ecosistemes que proporcionen serveis
essencials per a la supervivència i el benestar humans. No obstant això,
aquesta biodiversitat està disminuint ràpidament a causa d'activitats humanes
com la destrucció d'hàbitats, el canvi climàtic, les espècies invasores i les
malalties emergents. Aquests factors interconnectats estan provocant una pèrdua
generalitzada d'espècies i la ràpida degradació dels ecosistemes, cosa que
amenaça l'estabilitat ecològica global i la prosperitat humana. Per abordar
aquesta crisi es requereix un enfocament interdisciplinari que permeti
comprendre i mitigar els seus impactes, assegurant la preservació de la
biodiversitat i la sostenibilitat de les societats humanes.

En aquesta tesi desenvolupem mètodes teòrics i basats en dades per abordar
qüestions urgents en ecologia i biologia de la conservació a través de la
lent dels sistemes complexos i la investigació interdisciplinària.
Abordem una gamma de desafiaments contemporanis relacionats amb la pèrdua de
biodiversitat impulsada pel canvi climàtic i les malalties emergents.
Aquests desafiaments inclouen la propagació de malalties, l'acidificació dels
oceans i el declivi d'ecosistemes crítics com els esculls de corall i les
praderies marines. Ens basem en una combinació de models teòrics,
simulacions computacionals i tècniques avançades d'anàlisi de dades per
obtenir una comprensió més profunda daquests fenòmens ecològics complexos.

A les dues primeres parts d'aquesta tesi, desenvolupem models matemàtics de
propagació de malalties per omplir els buits de coneixement a la
dinàmica de transmissió de malalties de plantes marines i transmeses per
vectors. Ens centrem en dos exemples: l'Esdeveniment de Mortalitat Masiva (MME)
de \textit{Pinna nobilis} i les malalties de plantes transmeses per
vectors causats pel bacteri \textit{Xylella fastidiosa}. Investiguem
el paper de factors clau com la temperatura o la mobilitat de patògens a
la transmissió de l'MME i l'impacte de l'abundància estacional no periòdica
d'insectes vectors a la propagació de malalties de plantes. Aquests
models brinden perspectives sobre els mecanismes que impulsen la dinàmica de
aquestes malalties i el potencial per al seu control i maneig.

A la tercera part, apliquem aquest coneixement adquirit per desenvolupar un
nou marc teòric per predir la distribució potencial de malalties de
plantes transmeses per vectors en funció de factors ambientals i
climàtics. Demostrem la utilitat d'aquest model en predir el risc de la
malaltia de Pierce de la vinya, causada per \textit{Xylella fastidiosa}, en
escenaris climàtics actuals i futurs. La nostra metodologia representa un
avenç significatiu en el camp de la biogeografia de malalties, ja que
proporciona una forma d'integrar les interaccions ecològiques complexes
inherents a les malalties per predir el seu possible establiment a
funció de les condicions ambientals.

Finalment, a la quarta part d'aquesta tesi, desenvolupem i apliquem
mètodes basats en dades per monitoritzar i avaluar la salut dels
ecosistemes marins costaners. Presentem un marc innovador per reconstruir
dades absents en sèries temporals de pH oceànic utilitzant tècniques de
aprenentatge profund, que milloren la nostra capacitat de monitoritzar la
acidificació oceànica amb precisió. A més, fem servir aprenentatge automàtic
i imatges satel·litàries per mapejar i avaluar l'estat de les praderies
marines,
oferint un enfocament escalable i rendible per al monitoratge dels
ecosistemes. Finalment fem una anàlisi global de les propietats
espacials dels esculls de corall utilitzant dades de teledetecció,
descobrint patrons universals en la distribució de la mida i la geometria
dels esculls. Aquests coneixements són crucials per desenvolupar
estratègies de conservació específiques per protegir aquests ecosistemes
vulnerables.

Aquesta tesi subratlla la importància de la investigació interdisciplinària,
integrant la teoria ecològica, la ciència de sistemes complexos i la
intel·ligència artificial per abordar els desafiaments ecològics. Les troballes
contribueixen al desenvolupament d'estratègies de conservació efectives, amb el
objectiu de mitigar els impactes del canvi climàtic i les malalties
emergents a la biodiversitat. En última instància, aquest treball recolza els
esforços per preservar la integritat dels ecosistemes i garantir la
sostenibilitat de les societats humanes davant dels canvis ambientals en
curs.

\vfill