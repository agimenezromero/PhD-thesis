\pagebreak
\thispagestyle{empty}

\begin{center}
    \textbf{\Large Resumen}
\end{center}

La vida en la Tierra ha evolucionado a lo largo de millones de años,
produciendo una rica diversidad de especies y ecosistemas que
son fundamentales para la supervivencia humana. Sin embargo, esta biodiversidad
está disminuyendo rápidamente debido a actividades humanas como la destrucción
de hábitats, el cambio climático, las especies invasoras y las enfermedades
emergentes. Estos factores interconectados están provocando una pérdida
generalizada de especies y la degradación acelerada de los ecosistemas, lo que
amenaza la estabilidad ecológica global y la prosperidad humana. Para abordar
esta crisis se requiere un enfoque interdisciplinario que permita comprender y
mitigar sus impactos, asegurando la preservación de la biodiversidad y la
sostenibilidad de las sociedades humanas.

En esta tesis desarrollamos métodos teóricos y basados en datos para estudiar
problemas relacionados con la pérdida de biodiversidad causada por el cambio
climático y las enfermedades emergentes. A través de la lente de los sistemas
complejos, abordamos desafíos contemporáneos como la propagación de
enfermedades, la acidificación de los océanos y el declive de ecosistemas
críticos como los arrecifes de coral o las praderas marinas. Nos basamos en una
combinación de modelos matemáticos, simulaciones computacionales y técnicas
avanzadas de análisis de datos para obtener una comprensión más profunda de
estos complejos fenómenos ecológicos.

En las dos primeras partes de esta tesis, desarrollamos modelos matemáticos
para avanzar nuestra comprensión de la dinámica de transmisión en enfermedades
marinas y transmitidas por vectores. Nos centramos en dos ejemplos: el Evento
de Mortalidad Masiva (MME) de \textit{Pinna nobilis} y las enfermedades de
plantas causadas por la bacteria \textit{Xylella fastidiosa} (Xf). Investigamos
el papel de factores clave como la temperatura o la movilidad de patógenos en
la transmisión del MME y el impacto de la estacionalidad en la abundancia de
vectores en la propagación de enfermedades de plantas. Estos modelos brindan
nuevas perspectivas sobre los mecanismos que facilitan estas enfermedades asi
como sobre su control y manejo.

En la tercera parte, aplicamos este conocimiento adquirido para desarrollar un
nuevo marco teórico para predecir la distribución potencial de enfermedades de
plantas transmitidas por vectores en función de factores climáticos.
Demostramos la utilidad de este modelo al predecir el riesgo de la
enfermedad de Pierce de la vid, causada por Xf, en escenarios climáticos
actuales y futuros. Nuestra metodología representa un avance significativo en
el campo de la biogeografía de enfermedades, permitiendo integrar las complejas
interacciones ecológicas inherentes a las enfermedades para predecir su posible
establecimiento en función de las condiciones ambientales.

Finalmente desarrollamos métodos basados en datos para monitorizar y
evaluar la salud de los ecosistemas marinos costeros. Utilizando técnicas de
aprendizaje profundo, presentamos un marco innovador para reconstruir series
temporales de pH oceánico incompletas, mejorando nuestra capacidad de
monitorizar la acidificación oceánica con precisión. Además, empleamos
aprendizaje automático e imágenes satelitales para mapear y evaluar el estado
de las praderas marinas de \textit{Posidonia oceanica}, ofreciendo un enfoque
escalable y rentable para monitorizar estos ecosistemas. Finalmente
realizamos un análisis global de las propiedades espaciales de los arrecifes de
coral utilizando datos de teledetección, descubriendo patrones universales en
la distribución del tamaño y la geometría de los arrecifes.

Esta tesis subraya la importancia de la investigación interdisciplinar,
integrando ecología, sistemas complejos e inteligencia artificial para abordar
los desafíos ecológicos. Los hallazgos contribuyen al desarrollo de estrategias
de conservación, con el objetivo de mitigar los impactos del cambio climático y
las enfermedades emergentes. Nuestro trabajo contribuye a
preservar la integridad de los ecosistemas y garantizar la sostenibilidad de
las sociedades humanas frente al cambio climático.

\vfill