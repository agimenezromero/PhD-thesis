\pagebreak
\thispagestyle{empty}

\begin{center}
    \textbf{\Large Preface}
\end{center}

The thesis you are about to read (or probably just leaf through) is not a
traditional one. I did not follow a single well defined research line, there
was not a single well defined methodology, and I did not have a single well
defined goal. I don't think that this is inherently bad, and perhaps this
situation is indeed increasingly common, but yet we are asked to write a thesis
that follows a traditional structure and that tells a story that is not
necessarily the one we lived. Here I would like to briefly explain the factors
that led to the diversity of topics in this thesis.

The first one is of course my own personality: curious, open-minded... sorry,
this is just an insider joke, look at the IFISC lemma! Now seriously, my
curiosity, and perhaps my little patience, made me jump from one topic to
another, from one methodology to another, and from one goal to another.
Secondly, and less joyful, the lack of funding: I never obtained a grant for my
PhD. Thus, I had to constantly think on how to get funding for
the next year, even if not actively, as nothing could be taken for granted.
This made me go into each and almost every opportunity that appeared. But don't
get me wrong, I really like working on different topics, just that this does
not help to build a traditional thesis, precisely. Here I must say that I was
not alone in this situation, my supervisor had always several plans for
possible funding sources and he always gave me the freedom to choose on what to
work on. Sadly, this is not the case for many PhD students, with or without
grants. I am conscious that I was truly lucky on this matter. Finally, the
nature of the scientific system that has been built over the last decades:
publish or perish, impact factor, h-index, etc. I do not consider that I
have made ``bad'' science, but perhaps I would have continued doing research on
some topics that I abandoned if I had not been ``worried'' about the impact of
my work.

At any rate, throughout the pages of this thesis and especially in the
Introduction, I will try to convince you that all the work I have done is
connected, that there is a common thread that connects all the topics I have
worked on. Despite I believe that this is true (I am of course not lying to
you), if we are to be honest, this takes a secondary role.

To finish, just to comment that I have tried to write this thesis for all
audiences, not only for the experts in the field. I have tried to explain the
concepts in a simple way, and I have avoided the use of jargon as much as
possible, especially mathematical one. I have also tried to make it enjoyable
to read, with some more personal comments.

I hope you enjoy reading this thesis as much as I enjoyed when I saw it
finished (I wouldn't say ``as I enjoyed it writting it'', precisely) and that
you find it interesting and inspiring.

\vfill