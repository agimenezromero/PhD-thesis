\pagebreak
\thispagestyle{empty}

\begin{center}
    \textbf{\Large Agraïments}
\end{center}

Aquesta tesi no hauria estat possible sense l'ajuda de moltes persones. En
linia amb la longitud de la tesi, aquesta secció també serà llarga.

En primer lloc, la persona que més ha contribuit a aquest treball és el meu
director, el Dr. Manuel Matías. És gracies a ell que he tingut l'oportunitat de
realitzar aquesta tesi i començar la meva carrera investigadora. Li estic
especialment agraït per confiar en mi des del primer moment, quan el meu
historial acadèmic no era el més prometedor. També li estic agraït per la forma
en què m'ha \textbf{guiat} durant tot el procés i tot el que m'ha ensenyat,
tant en l'àmbit acadèmic com en el personal. Tinc clar que, per mi, es el
millor director que hauria pogut tenir.

Els meus colaboradors també mereixen un agraïment especial. Iris Hendriks, qui
va ser la primera persona amb qui vaig començar a colaborar, ja durant el meu
TFM, i Amalia Grau. Juntament amb el meu director conformen els coautors del
meu primer article sobre l'esdeveniment de mortalitat massiva de les nacres.
Cristóbal López i Federico Vazquez (a.k.a Fede), amb qui vam continuar
treballant en aquest tema. Seguim amb Eduardo Moralejo, coautor de la majoria
d'articles presents en aquesta tesi, qui em va endinsar en el món de la
fitopatologia i amb qui vaig començar a treballar en el projecte de la
\textit{Xylella fastidiosa}. Espero que aquesta col·laboració segueixi en el
futur. En relació amb aquest mateix tema també vull agrair a la Clara Lago,
Arantzazu Moreno i Alberto Fereres, amb qui he tingut l'oportunitat de
col·laborar i qui em van ``acollir'' durant el congrés que organitzaven a
Madrid. També agrair aqui a un científic ``de la casa'', Jose Ramasco. No només
per la part purament científica, sinó també per les converses que hem tingut.
Probablement és de les persones més ocupades del IFISC, però també de les més
maques. Per acabar amb aquest tema vull agrair a Jose Gutiérrez i Maialen
Iturbide, qui em van acollir durant la meva estada a Santander. No tinc
paraules per expressar la meva gratitud per la seva amabilitat, incloent
que m'oferissin el seu cotxe per visitar la zona! Va ser una experiència molt
enriquidora i espero que puguem seguir col·laborant en el futur.

Vull agrair a la meva família, per haver-me donat l'oportunitat de
formar-me i per haver-me donat suport en tot moment. També vull agrair a tots
els amics que m'han acompanyat durant aquests anys, per haver fet més fàcil
aquest camí.

% Clases: Pere, Rosa. 
% Coautors: Susana, Tomas, Dhafer, Duarte

\vfill