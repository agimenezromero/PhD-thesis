\pagebreak
\thispagestyle{empty}

\begin{center}
    \textbf{\Large Abstract}
\end{center}

Life on Earth has evolved over billions of years, resulting in a rich
diversity of species and ecosystems that provide essential services for
human survival and well-being. However, this biodiversity is rapidly
declining due to human activities such as habitat destruction, climate
change, invasive species and emergent diseases. These interconnected
drivers are causing widespread loss of species and degradation of
ecosystems, threatening global ecological stability and human prosperity.
Addressing this crisis requires an interdisciplinary approach to understand
and mitigate its impacts, ensuring the preservation of biodiversity and the
sustainability of human societies.

In this thesis we develop theoretical and data-driven methods
to address pressing issues in Ecology and Conservation Biology through the
lens of Complex Systems and interdisciplinary research. We address a
range of contemporary challenges related to biodiversity loss driven by
climate change and emerging diseases. These challenges include the spread
of diseases, ocean acidification and the decline of critical ecosystems such as
coral reefs and seagrass meadows. We rely on a combination of theoretical
models, computational simulations, and advanced data analysis techniques to
gain a deeper understanding of these complex ecological phenomena.

In the first two parts of this thesis we develop mathematical models of
disease spread to fill knowledge gaps in the transmission dynamics of
marine and vector-borne plant diseases. We focus on two case studies: the
Mass Mortality Event (MME) of \textit{Pinna nobilis} and the vector-borne plant
diseases caused by the bacterium \textit{Xylella fastidiosa}. We investigate
the role of key factors such as temperature or pathogen mobility in the
transmission of the MME, and the impact of the non-periodic seasonal abundance
of insect vectors on the spread of plant diseases. These models provide
insights into the mechanisms driving the dynamics of these diseases, and the
potential for their control and management.

In the third part, we apply this gained knowledge to develop a novel
theoretical framework to predict the potential distribution of
vector-borne plant diseases based on environmental and climatic factors. We
demonstrate the utility of this model by predicting the risk of Pierce's
Disease of grapevines, caused by \textit{Xylella fastidiosa}, under current
and future climate scenarios. Our methodology represents a significant
advancement in the field of disease biogeography, providing a way to
integrate the inherent complex ecological interactions of diseases to predict
its potential establishment based on environmental conditions.

Finally, in the fourth part of this thesis, we develop and apply
data-driven methods to monitor and assess the health of coastal marine
ecosystems. We present an innovative framework to reconstruct missing data
from ocean pH time-series using deep learning techniques, which enhance our
ability to monitor ocean acidification accurately. Additionally, we employ
machine learning and satellite imagery to map and evaluate the condition of
seagrass meadows, offering a scalable and cost-effective approach to
ecosystem monitoring. Moreover, we conduct a global analysis of the spatial
properties of coral reefs using remote sensing data, uncovering universal
patterns in reef size distribution and geometry. These insights are crucial
for developing targeted conservation strategies to protect these vulnerable
ecosystems.

This thesis underscores the importance of interdisciplinary research,
integrating ecological theory, complex systems science, and artificial
intelligence to tackle ecological challenges. The findings contribute to
the development of effective conservation strategies, aiming to mitigate
the impacts of climate change and emergent diseases on biodiversity.
Ultimately, this work supports efforts to preserve the integrity of
ecosystems and ensure the sustainability of human societies in the face of
ongoing environmental changes.

\vfill